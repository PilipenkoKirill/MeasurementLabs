%!TEX program = xelatex.exe
\documentclass[14pt,a4paper]{article}
% \usepackage[pdftex,
%     pdfauthor={К.С.~Пилипенко},
%     pdfsubject={The Subject},
%     pdfkeywords={Первое ключевое слово, второе ключевое слово},
%     pdfproducer={LuaLatex with hyperref},
%     pdfcreator={Lualatex},
%     % hidelinks
% ]{hyperref}
%%%%%%%%%%%%%%Пользовательские команды%%%%%%%%%
\usepackage[euler]{textgreek}
\usepackage{calc}
\usepackage{color}
\usepackage{listing}
\usepackage{svg}
\usepackage{hhline}
\usepackage{multirow}
\usepackage{latexsym,amsmath,amssymb,amsbsy,graphicx}
\usepackage{listings}
\usepackage{icomma}
\usepackage[obeyspaces]{url} %Позволяет прописывать путь к файлам
\usepackage[american,siunitx]{circuitikz}
\def\centerarc[#1][#2](#3:#4:#5){\draw[#1]($(#2)+({#5*cos(#3)},{#5*sin(#3)})$)}
\usepackage[version=4]{mhchem} % the canonical chemistry package (example: \ce{^{32}_{15}P})
\usepackage{graphicx}
\usepackage{pgfplots}
\usepackage{wrapfig}
\graphicspath{{images/}}
\DeclareGraphicsExtensions{.pdf,.png,.jpg}
\usepackage{hyperref}
\hypersetup{
    colorlinks=true,
    linkcolor=blue,
    filecolor=magenta,      
    urlcolor=cyan,
    pdfauthor={К.С.~Пилипенко},
    pdfsubject={Основы обработки результатов исследования в физике},
    % pdfkeywords={Первое ключевое слово, второе ключевое слово},
    pdfproducer={LuaLatex with hyperref},
    pdfcreator={Lualatex},
    pdfpagemode=FullScreen,
    }
\urlstyle{same}
%Форматирование разделов
\usepackage{titlesec}
\usepackage{multicol} %Многоколоночный список
%%% Кастомизация разделов
\titleformat
{\section} % command
[block] % shape
{\normalfont\bfseries\itshape} % format
{Задание №\thesection.}{0.5em}{} % label
\newcommand{\progress}{\begin{center}\large\bfseries Ход работы\end{center}}
\newcommand{\questions}{\begin{center}\large\bfseries  Контрольные вопросы \end{center}}
%%%%%%%%%%%%%%%%%%%%%%%%Оформление по ГОСТУ
\usepackage{fontspec}
\setmainfont[Renderer=Basic,Ligatures={TeX}]{Times New Roman}
% \usepackage[english,russian]{babel} %Поддержка русской локализации
\setmonofont{dejavusansmono} % verbatim с кириллицей
\usepackage[14pt]{extsizes} % для того чтобы задать нестандартный 14-ый размер шрифта
\usepackage{indentfirst} %Задаёт отступ самого первого абзаца
\setlength\parindent{1.25cm}
\usepackage[a4paper, left=3cm, top=1.5cm, right=1.5cm, bottom=2cm]{geometry}
\usepackage{setspace}
\sloppy %Выравнивание текст по ширине и решение проблемы переполнением строки
\onehalfspacing %Полуторный интервал
%%%%%%%%%%%%%%%%%%%%%%%%%%%%% Добавление списка литературы
\usepackage{polyglossia}[2014/05/21]            % Поддержка многоязычности (fontspec подгружается автоматически)
\setmainlanguage[babelshorthands=true]{russian}  % Язык по-умолчанию русский с поддержкой приятных команд пакета babel
\setotherlanguage{english}                       % Дополнительный язык = английский (в американской вариации по-умолчанию)
\setmonofont{Courier New}
\newfontfamily\cyrillicfonttt{Courier New}
    \defaultfontfeatures{Ligatures=TeX,Mapping=tex-text}
\setmainfont{Times New Roman}
\newfontfamily\cyrillicfont{Times New Roman}
\setsansfont{Arial}
\newfontfamily\cyrillicfontsf{Arial}
\usepackage[
    bibencoding=utf8,% кодировка bib файла
    sorting=none,% настройка сортировки списка литературы
    style=gost-numeric,% стиль цитирования и библиографии (по ГОСТ)
    language=autobib,% получение языка из babel/polyglossia, default: autobib % если ставить autocite или auto, то цитаты в тексте с указанием страницы, получат указание страницы на языке оригинала
    autolang=other,% многоязычная библиография
    clearlang=true,% внутренний сброс поля language, если он совпадает с языком из babel/polyglossia
    defernumbers=true,% нумерация проставляется после двух компиляций, зато позволяет выцеплять библиографию по ключевым словам и нумеровать не из большего списка
    sortcites=true,% сортировать номера затекстовых ссылок при цитировании (если в квадратных скобках несколько ссылок, то отображаться будут отсортированно, а не абы как)
    doi=false,% Показывать или нет ссылки на DOI
    isbn=false,% Показывать или нет ISBN
]{biblatex}
\DeclareSourcemap{ %модификация bib файла перед тем, как им займётся biblatex 
    \maps{
        \map{% перекидываем значения полей language в поля langid, которыми пользуется biblatex
            \step[fieldsource=language, fieldset=langid, origfieldval, final]
            \step[fieldset=language, null]
        }
    }
}
\emergencystretch=1em
\addbibresource{literature.bib}
%%%%%%%%%%%%%%%%%%%%%%%%
\author{\href{mailto:www-kirill.pilipenko@yandex.ru}{К.С.~Пилипенко} \href{https://github.com/PilipenkoKirill/MeasurementLabs}{\includegraphics[width=.5cm]{images/gitHubLogo.pdf}}} %Через \and можно добавить ещё авторов
\date{\the\year{}}

\usepackage{booktabs}
\usepackage{pgfplotstable}
\title{Лабораторная работа №7 \\ \textit{Анализ выборки. Критерий согласия Пирсона.}}
\begin{document}
\maketitle
Критерий согласия Пирсона (Хи-квадрат) был придуман для проверки значимости расхождения эмпирических (наблюдаемых) и теоретических (ожидаемых) частот. Выражается следующей формулой:
\begin{equation} \label{PoissonDistrib}
    \chi^2 = \sum\limits_i^n \frac{(O_i - E_i)^2}{E_i},
\end{equation}
где $O_i$~---~наблюдаемые частоты (Observed), $E_i$~---~ожидаемые частоты (Expected). 

Полученное значение $\chi^2$ сравнивают с теоретически рассчитанным критическим значением $\chi^2_\text{кр.}$, которое зависит от значения доверительной вероятности (как правило принимается равным 95\%) и числа степеней свободы $k$, которое на один меньше количества уникальных значений в выборке ($k = N - 1$). 

Для расчета критического значения критерия $\chi^2_\text{кр.}$ можно воспользоваться специальной таблицей, но лучше и проще всего воспользоваться функцией \texttt{ХИ2.ОБР.ПХ}

% Напомню, что в большинстве случаев распределение дискретной случайной величины ($N$) подчиняется формуле Пуассона:
% \begin{equation}
% 	P(N) = \frac{\bar{N}^N e^{-\bar{N}}}{N!},
% \end{equation}
% где $\bar{N}$~---~среднее значение. Для расчета этой вероятности можно воспользовать функцией \href{https://support.microsoft.com/ru-ru/office/%D1%84%D1%83%D0%BD%D0%BA%D1%86%D0%B8%D1%8F-%D0%BF%D1%83%D0%B0%D1%81%D1%81%D0%BE%D0%BD-%D1%80%D0%B0%D1%81%D0%BF-8fe148ff-39a2-46cb-abf3-7772695d9636}{\texttt{ПУАССОН.РАСП(x;среднее;интегральная)}}, 
% где 
% \verb=x=~---~значение, для которого строится распределение,
% \verb=среднее=~---~среднее арифметическое распределения,
% \verb=интегральная=~---~логическое значение, определяющее форму функции. Если \texttt{ИСТИНА} функция возвращает вероятность того, что количество событий будет меньше, чем N то есть $P(n<N)$, если \texttt{ЛОЖЬ} то, возвращается значение функции взвешенной вероятности, то есть вероятность того, что количество происходящих событий будет ровно $N$ раз. Все аргументы являются обязательными.

Для получения нормального распределения используется функция \texttt{НОРМ.РАСП($x$;$\bar{x}$;$\sigma$;интегральная)}, где \\
	$x$~---~значение, для которого строится распределение. \\
    $\bar{x}$~---~среднее арифметическое распределения.\\
    $\sigma$~---~стандартное отклонение распределения.\\
    \texttt{интегральная}~---~логическое значение, определяющее форму функции. Если \texttt{ИСТИНА} функция возвращает значение функции распределения $f(x)$, если \texttt{ЛОЖЬ} то, возвращается значение функции плотности вероятности $\omega(x)$.

Последний аргумент является необязательным и по умолчанию стоит как \texttt{ЛОЖЬ}. Более подробно о синтаксисе этой функции можно узнать \href{https://support.microsoft.com/ru-ru/office/%D1%84%D1%83%D0%BD%D0%BA%D1%86%D0%B8%D1%8F-%D0%BD%D0%BE%D1%80%D0%BC-%D1%80%D0%B0%D1%81%D0%BF-edb1cc14-a21c-4e53-839d-8082074c9f8d}{здесь}.

\progress{}

\section{Генерация выборки}
\noindent\begin{itemize}
	% \item Сгенерируйте выборку с помощью программы \texttt{NormDistGen.exe}. После запуска программы в той же папке появится новый файл;
	\item Создайте файл с расширением \texttt{.xlsm}. По умолчанию вкладка \texttt{Разработчик} спрятана, поэтому надо её включить. Заходим по пути \texttt{Файл$\rightarrow$Параметры$\rightarrow$Настроить ленту} и ставим галочку на вкладке \texttt{Разработчик};
	\item В разделе \texttt{Разработчик} нажмите на \texttt{макросы}, дайте любое название макроса и создайте его. Далее в открывшемся окне введите код из листинга \ref{NormDistGen};
	\lstset{ %
	language=[Visual]Basic,
	keywordstyle=\color{brown},          % Стиль ключевых слов
	breaklines=true,
	commentstyle=\color{gray}, % white comments
	numbers=left, 
	backgroundcolor=\color{lightgray!5!white},
	tabsize=4,
	texcl=true
	} 
	\begin{lstlisting} 
	Sub NormDistGen()
		' Сюда надо поместить код из листинга
	End Sub  
	\end{lstlisting}
	\item При запуске кода в первом столбце таблицы появится сгенерированная выборка объёмом 500 элементов;
\end{itemize}
\section{Получение наблюдаемых и ожидаемых частот}	
\begin{itemize}
	\item Найдём максимальный и минимальный элементы выборки с помощью функций МАКС и МИН соответственно. 
	\item Создайте столбец из целых чисел, значения которого будут использованы в качестве границ интервалов ($N$). Первое и последнее числа столбца должны быть больше минимального и максимального значения выборки соответственно. 
	
	\textit{Пример. Если $x_{min} = -0,961$ и $x_{max} = 25,983$ то получится столбец от 0 до 26 с шагом 1};

	\item Следующий шаг~---~получить частоты попадания элементов выборки в эти интервалы. Для этого необходимо воспользоваться функцией \href{https://support.microsoft.com/ru-ru/office/%D1%84%D1%83%D0%BD%D0%BA%D1%86%D0%B8%D1%8F-%D1%87%D0%B0%D1%81%D1%82%D0%BE%D1%82%D0%B0-44e3be2b-eca0-42cd-a3f7-fd9ea898fdb9}{\texttt{ЧАСТОТА(массив данных; массив верхних границ диапазонов)}}. Перед тем как задавать функцию нужно выделить диапазон ячеек, размер которого соответствует размеру массива границ. После нажатия комбинации \texttt{CTRL+SHIFT+ENTER} сформируется массив частот $O(N)$;
	\item Постройте график зависимости массива частот $O$ от массива границ $N$;
	\item Теперь нужно получить столбец $E(N)$. Перед этим нужно посчитать столбец вероятностей  ($P_i$) для каждого $N$ (значения из массива границ) с помощью функции \texttt{НОРМ.РАСП} с пустым последним параметром. Остаётся открытым вопрос, откуда брать среднее ($\bar{N}$) и $\sigma$? Эти параметры можно подобрать вручную построив два графика в одних координатах $E(N)$ и $O(N)$, и добившись их наилучшего соответствия;
	\item Чтобы найти ожидаемые (теоретические) частоты $E_i$ нужно умножить соответствующие вероятности ($P_i$) на объём выборки, то есть на 500. Ож. частоты $E_i$ должны быть целыми ненулевыми числами, поэтому стоит использовать функцию \texttt{ОКРУГЛВВЕРХ};
\end{itemize}
\section{Оценка критерия Пирсона}
\begin{itemize}
	\item Используя формулу \ref{PoissonDistrib} посчитать критерий Пирсона;
	\item Используя функцию \texttt{ХИ2.РАСП} постройте график плотности распределения $\omega_{\chi_\text{кр.}^2}(N)$ (последний параметр должен быть \verb=ЛОЖЬ=);
	\item В отдельном столбце посчитать критическое значение пользуясь функцией \texttt{ХИ2.ОБР.ПХ}. Указать это значение на графике $\omega(\chi_\text{кр.}^2)$. Сравнить полученное значение с экспериментальным;
	\item Получить p-value используя функцию \texttt{ХИ2.РАСП.ПХ} для посчитанного $\chi^2$
	\item Чтобы убедится в правильности полученного результата сравните p-value со значением функции \texttt{ХИ2.ТЕСТ}, которая принимает на вход массивы наблюдаемых и ожидаемых частот.    
\end{itemize}

\questions{}
\begin{enumerate}
	\item Сформулируйте нулевую гипотезу $H_0$. Назовите условие, при котором можно отклонить нулевую гипотезу.
	% \item Вывести дисперсию распределение Пуассона (D[N]).
	\item Можно ли в программе Excel сгенерировать выборку имеющую нормальное распределение? Если можно, то как это реализовать? %=НОРМОБР(СЛЧИС();сред.;стоткл)
	\item Что такое p-value (p-значение)? Чему численно равно это значение?
	\item В каких случаях используется функция \texttt{ХИ2.ОБР}? Что она позволяет оценить
\end{enumerate}
\begin{center}
	\textbf{\Large Приложение}
\end{center}
\lstset{ %
	language=[Visual]Basic,
	keywordstyle=\color{brown},          % Стиль ключевых слов
	breaklines=true,
	numbers=left, 
	backgroundcolor=\color{lightgray!5!white},
	tabsize=4,
	label=NormDistGen, 
	caption={Код генератора выборки с нормальным распределением}
	} 
	\renewcommand{\lstlistingname}{Листинг}
\begin{lstlisting}
	Randomize
	Dim i As Long
	Dim mean As Integer
	Dim sigma As Integer
	Dim random As Double

	mean = 10 * Rnd + 6
	sigma = 3 * Rnd + 2
	i = 501
	Range("A1").Select
	For i = 2 To i
		random = WorksheetFunction.NormInv(Rnd, mean, sigma)
		ActiveCell.Value = random
		ActiveCell.Offset(1, 0).Select
	Next i
	\end{lstlisting}
\end{document}