%!TEX program = xelatex.exe
\documentclass[14pt,a4paper]{article}
% \usepackage[pdftex,
%     pdfauthor={К.С.~Пилипенко},
%     pdfsubject={The Subject},
%     pdfkeywords={Первое ключевое слово, второе ключевое слово},
%     pdfproducer={LuaLatex with hyperref},
%     pdfcreator={Lualatex},
%     % hidelinks
% ]{hyperref}
%%%%%%%%%%%%%%Пользовательские команды%%%%%%%%%
\usepackage[euler]{textgreek}
\usepackage{calc}
\usepackage{color}
\usepackage{listing}
\usepackage{svg}
\usepackage{hhline}
\usepackage{multirow}
\usepackage{latexsym,amsmath,amssymb,amsbsy,graphicx}
\usepackage{listings}
\usepackage{icomma}
\usepackage[obeyspaces]{url} %Позволяет прописывать путь к файлам
\usepackage[american,siunitx]{circuitikz}
\def\centerarc[#1][#2](#3:#4:#5){\draw[#1]($(#2)+({#5*cos(#3)},{#5*sin(#3)})$)}
\usepackage[version=4]{mhchem} % the canonical chemistry package (example: \ce{^{32}_{15}P})
\usepackage{graphicx}
\usepackage{pgfplots}
\usepackage{wrapfig}
\graphicspath{{images/}}
\DeclareGraphicsExtensions{.pdf,.png,.jpg}
\usepackage{hyperref}
\hypersetup{
    colorlinks=true,
    linkcolor=blue,
    filecolor=magenta,      
    urlcolor=cyan,
    pdfauthor={К.С.~Пилипенко},
    pdfsubject={Основы обработки результатов исследования в физике},
    % pdfkeywords={Первое ключевое слово, второе ключевое слово},
    pdfproducer={LuaLatex with hyperref},
    pdfcreator={Lualatex},
    pdfpagemode=FullScreen,
    }
\urlstyle{same}
%Форматирование разделов
\usepackage{titlesec}
\usepackage{multicol} %Многоколоночный список
%%% Кастомизация разделов
\titleformat
{\section} % command
[block] % shape
{\normalfont\bfseries\itshape} % format
{Задание №\thesection.}{0.5em}{} % label
\newcommand{\progress}{\begin{center}\large\bfseries Ход работы\end{center}}
\newcommand{\questions}{\begin{center}\large\bfseries  Контрольные вопросы \end{center}}
%%%%%%%%%%%%%%%%%%%%%%%%Оформление по ГОСТУ
\usepackage{fontspec}
\setmainfont[Renderer=Basic,Ligatures={TeX}]{Times New Roman}
% \usepackage[english,russian]{babel} %Поддержка русской локализации
\setmonofont{dejavusansmono} % verbatim с кириллицей
\usepackage[14pt]{extsizes} % для того чтобы задать нестандартный 14-ый размер шрифта
\usepackage{indentfirst} %Задаёт отступ самого первого абзаца
\setlength\parindent{1.25cm}
\usepackage[a4paper, left=3cm, top=1.5cm, right=1.5cm, bottom=2cm]{geometry}
\usepackage{setspace}
\sloppy %Выравнивание текст по ширине и решение проблемы переполнением строки
\onehalfspacing %Полуторный интервал
%%%%%%%%%%%%%%%%%%%%%%%%%%%%% Добавление списка литературы
\usepackage{polyglossia}[2014/05/21]            % Поддержка многоязычности (fontspec подгружается автоматически)
\setmainlanguage[babelshorthands=true]{russian}  % Язык по-умолчанию русский с поддержкой приятных команд пакета babel
\setotherlanguage{english}                       % Дополнительный язык = английский (в американской вариации по-умолчанию)
\setmonofont{Courier New}
\newfontfamily\cyrillicfonttt{Courier New}
    \defaultfontfeatures{Ligatures=TeX,Mapping=tex-text}
\setmainfont{Times New Roman}
\newfontfamily\cyrillicfont{Times New Roman}
\setsansfont{Arial}
\newfontfamily\cyrillicfontsf{Arial}
\usepackage[
    bibencoding=utf8,% кодировка bib файла
    sorting=none,% настройка сортировки списка литературы
    style=gost-numeric,% стиль цитирования и библиографии (по ГОСТ)
    language=autobib,% получение языка из babel/polyglossia, default: autobib % если ставить autocite или auto, то цитаты в тексте с указанием страницы, получат указание страницы на языке оригинала
    autolang=other,% многоязычная библиография
    clearlang=true,% внутренний сброс поля language, если он совпадает с языком из babel/polyglossia
    defernumbers=true,% нумерация проставляется после двух компиляций, зато позволяет выцеплять библиографию по ключевым словам и нумеровать не из большего списка
    sortcites=true,% сортировать номера затекстовых ссылок при цитировании (если в квадратных скобках несколько ссылок, то отображаться будут отсортированно, а не абы как)
    doi=false,% Показывать или нет ссылки на DOI
    isbn=false,% Показывать или нет ISBN
]{biblatex}
\DeclareSourcemap{ %модификация bib файла перед тем, как им займётся biblatex 
    \maps{
        \map{% перекидываем значения полей language в поля langid, которыми пользуется biblatex
            \step[fieldsource=language, fieldset=langid, origfieldval, final]
            \step[fieldset=language, null]
        }
    }
}
\emergencystretch=1em
\addbibresource{literature.bib}
%%%%%%%%%%%%%%%%%%%%%%%%
\author{\href{mailto:www-kirill.pilipenko@yandex.ru}{К.С.~Пилипенко} \href{https://github.com/PilipenkoKirill/MeasurementLabs}{\includegraphics[width=.5cm]{images/gitHubLogo.pdf}}} %Через \and можно добавить ещё авторов
\date{\the\year{}}

\title{Лабораторная работа №9 \\ \textit{Построение графиков и символические вычисления в Mathcad}}
\begin{document}
\maketitle

\progress{}
\section{Работа с 2D-графиками}
\begin{enumerate}
	\item Постройте параметрический график \\
	$x=\sin{t}, \quad y = \cos{t}$
	\item Постройте параметрический график \\
	$x=16\sin^3t$ \\
	$y=13\cos{t} - 5\cos{2t} - 2\cos{3t} - \cos{4t}$
	\item Подберите параметрические уравнения и постройте график спирали в декартовой системе координат. \textbf{Подсказка: задайте положительную область определения и модифицируйте параметрический график из пункта 1}\\
	\begin{tikzpicture}
		\begin{axis}[
			trig format plots=rad,
			axis equal,
			grid = major,
		]
		\addplot [smooth, domain=0:16*pi, samples=200, orange] ({x*sin(x)}, {x*cos(x)});
		\end{axis}
	\end{tikzpicture}
	\item Построить в полярной системе координат график функции $R(\phi) = a·\cos{(m·\phi)}$, где $a = 2$, $m = 4$, $\phi$ меняется в пределах от $-π$ до $+π$ с шагом $0,01·π$.
\end{enumerate}

\section{Работа с 3D-графиками}
\noindent Построить графики
\begin{enumerate}
	\item $f(x,y) = sin(x^2+y^2)$, $x,y \in [-1;1]$
	\item $X:=\sin \left(\alpha\right) \cdot \cos \left(\beta\right) \quad Y:=\sin \left(\alpha\right) \cdot \sin \left(\beta\right) \quad Z:=\cos \left(\alpha\right)$\\
	$\alpha \in [0;\pi]$, $\beta \in [0;\pi]$
	\item $X=\cos{(u)}(\cos{(v)}+3)$ \\
	$Y=\sin{(u)}(\cos{(v)}+3)$ \\
	$Z=\sin{(v)}$ \\
	$u\in [−\pi;\pi],\quad v \in [−\pi;\pi]$
\end{enumerate}

\section{Анализ функций}
\begin{enumerate}
	\item Упростите выражение $\frac{\frac{(x-2 y)^2+1}{\left(2-\left(x+\frac{y}{x}+1\right)^2\right)}}{x}$.
	\item Раскройте скобки и приведите подобные члены в выражении $\left(\frac{\left(x^2-1\right)}{x+2}+3\right)^2$.
	\item Разложите на множители выражение $a^{8}-16$.
	\item Получить относительно $x$ массив полиномиальных коэффициентов \\ $(x+y)^4$.
\end{enumerate}

\noindent  Найти следующие интегралы с помощью опции solve
\begin{multicols}{2}
	\begin{enumerate}
		\item $\int\limits_1^y \frac{\sin x}{\sin x+\cos x} \cdot d x;$
		\item $\int\limits_0^a x \cdot(\sin x)^2 \cdot d x$
	\end{enumerate}
\end{multicols}
\noindent Представить следующие функции в виде разложения в ряд Тейлора (ключевое слово \texttt{series}), графически показать, при каких значениях аргумента $x$ аппроксимация многочленом является удовлетворительной:
\begin{multicols}{2}
	\begin{enumerate}
		\item $\sqrt[3]{x^2+2};$ 
		\item $\sqrt{x+\sin{x}}$
	\end{enumerate}
\end{multicols}

\nocite{Plis2003,Krestlev2010,MathCADshortcuts,Ochkov2016}
\printbibliography[title={Рекомендуемая литература}]
\end{document}