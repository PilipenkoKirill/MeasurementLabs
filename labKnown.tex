%!TEX program = xelatex.exe
\documentclass[14pt,a4paper]{article}
% \usepackage[pdftex,
%     pdfauthor={К.С.~Пилипенко},
%     pdfsubject={The Subject},
%     pdfkeywords={Первое ключевое слово, второе ключевое слово},
%     pdfproducer={LuaLatex with hyperref},
%     pdfcreator={Lualatex},
%     % hidelinks
% ]{hyperref}
%%%%%%%%%%%%%%Пользовательские команды%%%%%%%%%
\usepackage[euler]{textgreek}
\usepackage{calc}
\usepackage{color}
\usepackage{listing}
\usepackage{svg}
\usepackage{hhline}
\usepackage{multirow}
\usepackage{latexsym,amsmath,amssymb,amsbsy,graphicx}
\usepackage{listings}
\usepackage{icomma}
\usepackage[obeyspaces]{url} %Позволяет прописывать путь к файлам
\usepackage[american,siunitx]{circuitikz}
\def\centerarc[#1][#2](#3:#4:#5){\draw[#1]($(#2)+({#5*cos(#3)},{#5*sin(#3)})$)}
\usepackage[version=4]{mhchem} % the canonical chemistry package (example: \ce{^{32}_{15}P})
\usepackage{graphicx}
\usepackage{pgfplots}
\usepackage{wrapfig}
\graphicspath{{images/}}
\DeclareGraphicsExtensions{.pdf,.png,.jpg}
\usepackage{hyperref}
\hypersetup{
    colorlinks=true,
    linkcolor=blue,
    filecolor=magenta,      
    urlcolor=cyan,
    pdfauthor={К.С.~Пилипенко},
    pdfsubject={Основы обработки результатов исследования в физике},
    % pdfkeywords={Первое ключевое слово, второе ключевое слово},
    pdfproducer={LuaLatex with hyperref},
    pdfcreator={Lualatex},
    pdfpagemode=FullScreen,
    }
\urlstyle{same}
%Форматирование разделов
\usepackage{titlesec}
\usepackage{multicol} %Многоколоночный список
%%% Кастомизация разделов
\titleformat
{\section} % command
[block] % shape
{\normalfont\bfseries\itshape} % format
{Задание №\thesection.}{0.5em}{} % label
\newcommand{\progress}{\begin{center}\large\bfseries Ход работы\end{center}}
\newcommand{\questions}{\begin{center}\large\bfseries  Контрольные вопросы \end{center}}
%%%%%%%%%%%%%%%%%%%%%%%%Оформление по ГОСТУ
\usepackage{fontspec}
\setmainfont[Renderer=Basic,Ligatures={TeX}]{Times New Roman}
% \usepackage[english,russian]{babel} %Поддержка русской локализации
\setmonofont{dejavusansmono} % verbatim с кириллицей
\usepackage[14pt]{extsizes} % для того чтобы задать нестандартный 14-ый размер шрифта
\usepackage{indentfirst} %Задаёт отступ самого первого абзаца
\setlength\parindent{1.25cm}
\usepackage[a4paper, left=3cm, top=1.5cm, right=1.5cm, bottom=2cm]{geometry}
\usepackage{setspace}
\sloppy %Выравнивание текст по ширине и решение проблемы переполнением строки
\onehalfspacing %Полуторный интервал
%%%%%%%%%%%%%%%%%%%%%%%%%%%%% Добавление списка литературы
\usepackage{polyglossia}[2014/05/21]            % Поддержка многоязычности (fontspec подгружается автоматически)
\setmainlanguage[babelshorthands=true]{russian}  % Язык по-умолчанию русский с поддержкой приятных команд пакета babel
\setotherlanguage{english}                       % Дополнительный язык = английский (в американской вариации по-умолчанию)
\setmonofont{Courier New}
\newfontfamily\cyrillicfonttt{Courier New}
    \defaultfontfeatures{Ligatures=TeX,Mapping=tex-text}
\setmainfont{Times New Roman}
\newfontfamily\cyrillicfont{Times New Roman}
\setsansfont{Arial}
\newfontfamily\cyrillicfontsf{Arial}
\usepackage[
    bibencoding=utf8,% кодировка bib файла
    sorting=none,% настройка сортировки списка литературы
    style=gost-numeric,% стиль цитирования и библиографии (по ГОСТ)
    language=autobib,% получение языка из babel/polyglossia, default: autobib % если ставить autocite или auto, то цитаты в тексте с указанием страницы, получат указание страницы на языке оригинала
    autolang=other,% многоязычная библиография
    clearlang=true,% внутренний сброс поля language, если он совпадает с языком из babel/polyglossia
    defernumbers=true,% нумерация проставляется после двух компиляций, зато позволяет выцеплять библиографию по ключевым словам и нумеровать не из большего списка
    sortcites=true,% сортировать номера затекстовых ссылок при цитировании (если в квадратных скобках несколько ссылок, то отображаться будут отсортированно, а не абы как)
    doi=false,% Показывать или нет ссылки на DOI
    isbn=false,% Показывать или нет ISBN
]{biblatex}
\DeclareSourcemap{ %модификация bib файла перед тем, как им займётся biblatex 
    \maps{
        \map{% перекидываем значения полей language в поля langid, которыми пользуется biblatex
            \step[fieldsource=language, fieldset=langid, origfieldval, final]
            \step[fieldset=language, null]
        }
    }
}
\emergencystretch=1em
\addbibresource{literature.bib}
%%%%%%%%%%%%%%%%%%%%%%%%
\author{\href{mailto:www-kirill.pilipenko@yandex.ru}{К.С.~Пилипенко} \href{https://github.com/PilipenkoKirill/MeasurementLabs}{\includegraphics[width=.5cm]{images/gitHubLogo.pdf}}} %Через \and можно добавить ещё авторов
\date{\the\year{}}

\usepackage{pgfplotstable}
\usepackage{booktabs}
\pgfplotstableset{
every head row/.style={before row=\toprule,after row=\midrule},
every last row/.style={after row=\bottomrule}}
\title{Лабораторная работа №8 \\ \textit{Проверка равенства двух независимых выборок. Критерий Стьюдента}}
\begin{document}
\maketitle
Критерий согласия Пирсона (Хи-квадрат) был придуман для проверки значимости расхождения эмпирических (наблюдаемых) и теоретических (ожидаемых) частот. Выражается следующей формулой:
\begin{equation} \label{PoissonDistrib}
    \chi^2 = \sum\limits_i^n \frac{(O_i - E_i)^2}{E_i},
\end{equation}
где $O_i$~---~наблюдаемые частоты (Observed), $E_i$~---~ожидаемые частоты (Expected). 

Полученное значение $\chi^2$ сравнивают с теоретически рассчитанным критическим значением $\chi^2_\text{кр.}$, которое зависит от значения доверительной вероятности (как правило принимается равным 95\%) и числа степеней свободы $k$, которое на один меньше количества уникальных значений в выборке ($k = N - 1$). 

Для расчета критического значения критерия $\chi^2_\text{кр.}$ можно воспользоваться специальной таблицей, но лучше и проще всего воспользоваться функцией \texttt{ХИ2.ОБР.ПХ}

Напомню, что в большинстве случаев распределение дискретной случайной величины ($N$) подчиняется формуле Пуассона:
\begin{equation}
	P(N) = \frac{\bar{N}^N e^{-\bar{N}}}{N!},
\end{equation}
где $\bar{N}$~---~среднее значение. Для расчета этой вероятности можно воспользовать функцией \href{https://support.microsoft.com/ru-ru/office/%D1%84%D1%83%D0%BD%D0%BA%D1%86%D0%B8%D1%8F-%D0%BF%D1%83%D0%B0%D1%81%D1%81%D0%BE%D0%BD-%D1%80%D0%B0%D1%81%D0%BF-8fe148ff-39a2-46cb-abf3-7772695d9636}{\texttt{ПУАССОН.РАСП(x;среднее;интегральная)}}, 
где 
\verb=x=~---~значение, для которого строится распределение,
\verb=среднее=~---~среднее арифметическое распределения,
\verb=интегральная=~---~логическое значение, определяющее форму функции. Если \verb=ИСТИНА= функция возвращает вероятность $P(x)$, если \verb=ЛОЖЬ= то, возвращается значение функции взвешенной вероятности, то есть вероятность того, что количество происходящих событий будет ровно $N$ раз. Все аргументы являются обязательными.

% \verb=НОРМ.РАСП(x;среднее;стандартное_откл;интегральная)=, где \\
%     \verb=x=~---~значение, для которого строится распределение. \\
%     \verb=среднее=~---~среднее арифметическое распределения.\\
%     \verb=стандартное_откл=~---~стандартное отклонение распределения.\\
%     \verb=интегральная=~---~логическое значение, определяющее форму функции. Если \verb=ИСТИНА= функция возвращает значение функции распределения $f(x)$, если \verb=ЛОЖЬ= то, возвращается значение функции плотности вероятности $\omega(x)$.

% Все аргументы являются обязательными. Более подробно о синтаксисе этой функции можно узнать \href{https://support.microsoft.com/ru-ru/office/%D1%84%D1%83%D0%BD%D0%BA%D1%86%D0%B8%D1%8F-%D0%BD%D0%BE%D1%80%D0%BC-%D1%80%D0%B0%D1%81%D0%BF-edb1cc14-a21c-4e53-839d-8082074c9f8d}{здесь}.

\progress{}

\noindent\begin{minipage}{.85\textwidth}
\section{Проверить выборку на соответствие распределению Пуассона}
\noindent\begin{itemize}
	\item Выделите текст таблицы справа, скопируйте и вставьте в пустое поле таблицы Excel;
	\item Далее, кликнув по первому столбцу вставленного текста, переходим во вкладку «Данные». Там, в группе инструментов «Работа с данными» кликаем по кнопке «Текст по столбцам»; 
	\item В открывшемся окне нажимаем <<Далее>>, в списке <<Символов-разделителей>> кликаем на чекбокс <<пробел>>, далее нажимаем <<Готово>>.
	\item Следующий шаг - получить частоты распределения  Пуассона ($E_i$). Для этого необходимо воспользоваться функцией \texttt{ПУАССОН.РАСП}. Нужно рассчитать вероятность ($P_i$) для каждого $N$ и получить столбец вероятностей. Остаётся открытым вопрос, откуда брать \verb=среднее= ($\bar{N}$)? Его можно подобрать вручную построив два графика в одних координатах  $E_i(N)$ и $O_i(N)$, и добившись их наилучшего соответствия;
\end{itemize}
\end{minipage}
\begin{minipage}{.1\textwidth}
\pgfplotstabletypeset[
	col sep=comma,
	columns/0/.style={column name={N}},
	columns/1/.style={column name={$O_i$}},
]{PoissonDistrib.csv}
\end{minipage}
\begin{itemize}
	\item Чтобы найти ожидаемые (теоретические) частоты ($E_i$) нужно умножить соответствующие вероятности ($P_i$) на объём выборки, то есть на $\sum\limits_i^N O_i$;
	\item Используя формулу \ref{PoissonDistrib} посчитать критерий Пирсона (увы, функции Excel для этой формулы нет);
	\item Используя функцию \texttt{ХИ2.РАСП} постройте график плотности распределения $\omega(\chi_\text{кр.}^2)$ (последний параметр должен быть \verb=ЛОЖЬ=);
	\item В отдельном столбце посчитать критическое значение пользуясь функцией \texttt{ХИ2.ОБР.ПХ}. Указать это значение на графике $\omega(\chi_\text{кр.}^2)$. Сравнить полученное значение с экспериментальным;
	\item Получить p-value используя функцию \texttt{ХИ2.РАСП.ПХ} для посчитанного $\chi^2$
\end{itemize}

% \pgfplotstabletypeset[
% col sep=comma,
% % columns/inde/.append style={
%     string type,
%     string replace*={.}{,},
% 		% column name = {\textbf{C}${}_\mathbf{P}$},
% 	% },
% columns={A,B,A,B},
% display columns/0/.style={select equal part entry of={0}{2}},% first part of `A'
% display columns/1/.style={select equal part entry of={1}{2}},% first part of `B'
% % display columns/2/.style={select equal part entry of={1}{2}},% second part of `A'
% % display columns/3/.style={select equal part entry of={1}{2}},% second part of `B'
% ]
% {data.csv}

% \begin{center} % Клатрат SnPI
% 	\pgfplotstabletypeset[
% 	mytable,
% 	expheatcapfive,
% 	col sep=comma,
% 	columns/C/.append style={
%     string type,
%     string replace*={.}{,},
% 		% divide by = 4461.58954, %Перевод значений из молярной теплоёмкости в удельную
% 		column name = {\textbf{C}${}_\mathbf{P}$},
% 	},
% 	begin table=\begin{longtable},
% 	every head row/.append style={
% 			before row={ \caption{Молярная теплоёмкость клатрата \snpi{} $\mathrm{C_P(T)}$, Дж/(моль$\cdot$К)} \label{ExpValHeatCapSnPIAppendix} \\ \hline}%
% 	},
% 	end table=\end{longtable}
% 	]{Dissertation/chapter3/images/MolHeatCapSn24P193I8.csv}
% \end{center}

\questions{}
\begin{enumerate}
	\item Сформулируйте нулевую гипотезу $H_0$. Назовите условие, при котором можно отклонить нулевую гипотезу.
	\item Вывести дисперсию распределение Пуассона (D[N]).
	\item Что такое p-value (p-значение)? Чему численно равно это значение?
	\item В каких случаях используется функция \texttt{ХИ2.ОБР}? Что она позволяет оценить?
\end{enumerate}
\end{document}