%!TEX program = xelatex.exe
\documentclass[14pt,a4paper]{article}
% \usepackage[pdftex,
%     pdfauthor={К.С.~Пилипенко},
%     pdfsubject={The Subject},
%     pdfkeywords={Первое ключевое слово, второе ключевое слово},
%     pdfproducer={LuaLatex with hyperref},
%     pdfcreator={Lualatex},
%     % hidelinks
% ]{hyperref}
%%%%%%%%%%%%%%Пользовательские команды%%%%%%%%%
\usepackage[euler]{textgreek}
\usepackage{calc}
\usepackage{color}
\usepackage{listing}
\usepackage{svg}
\usepackage{hhline}
\usepackage{multirow}
\usepackage{latexsym,amsmath,amssymb,amsbsy,graphicx}
\usepackage{listings}
\usepackage{icomma}
\usepackage[obeyspaces]{url} %Позволяет прописывать путь к файлам
\usepackage[american,siunitx]{circuitikz}
\def\centerarc[#1][#2](#3:#4:#5){\draw[#1]($(#2)+({#5*cos(#3)},{#5*sin(#3)})$)}
\usepackage[version=4]{mhchem} % the canonical chemistry package (example: \ce{^{32}_{15}P})
\usepackage{graphicx}
\usepackage{pgfplots}
\usepackage{wrapfig}
\graphicspath{{images/}}
\DeclareGraphicsExtensions{.pdf,.png,.jpg}
\usepackage{hyperref}
\hypersetup{
    colorlinks=true,
    linkcolor=blue,
    filecolor=magenta,      
    urlcolor=cyan,
    pdfauthor={К.С.~Пилипенко},
    pdfsubject={Основы обработки результатов исследования в физике},
    % pdfkeywords={Первое ключевое слово, второе ключевое слово},
    pdfproducer={LuaLatex with hyperref},
    pdfcreator={Lualatex},
    pdfpagemode=FullScreen,
    }
\urlstyle{same}
%Форматирование разделов
\usepackage{titlesec}
\usepackage{multicol} %Многоколоночный список
%%% Кастомизация разделов
\titleformat
{\section} % command
[block] % shape
{\normalfont\bfseries\itshape} % format
{Задание №\thesection.}{0.5em}{} % label
\newcommand{\progress}{\begin{center}\large\bfseries Ход работы\end{center}}
\newcommand{\questions}{\begin{center}\large\bfseries  Контрольные вопросы \end{center}}
%%%%%%%%%%%%%%%%%%%%%%%%Оформление по ГОСТУ
\usepackage{fontspec}
\setmainfont[Renderer=Basic,Ligatures={TeX}]{Times New Roman}
% \usepackage[english,russian]{babel} %Поддержка русской локализации
\setmonofont{dejavusansmono} % verbatim с кириллицей
\usepackage[14pt]{extsizes} % для того чтобы задать нестандартный 14-ый размер шрифта
\usepackage{indentfirst} %Задаёт отступ самого первого абзаца
\setlength\parindent{1.25cm}
\usepackage[a4paper, left=3cm, top=1.5cm, right=1.5cm, bottom=2cm]{geometry}
\usepackage{setspace}
\sloppy %Выравнивание текст по ширине и решение проблемы переполнением строки
\onehalfspacing %Полуторный интервал
%%%%%%%%%%%%%%%%%%%%%%%%%%%%% Добавление списка литературы
\usepackage{polyglossia}[2014/05/21]            % Поддержка многоязычности (fontspec подгружается автоматически)
\setmainlanguage[babelshorthands=true]{russian}  % Язык по-умолчанию русский с поддержкой приятных команд пакета babel
\setotherlanguage{english}                       % Дополнительный язык = английский (в американской вариации по-умолчанию)
\setmonofont{Courier New}
\newfontfamily\cyrillicfonttt{Courier New}
    \defaultfontfeatures{Ligatures=TeX,Mapping=tex-text}
\setmainfont{Times New Roman}
\newfontfamily\cyrillicfont{Times New Roman}
\setsansfont{Arial}
\newfontfamily\cyrillicfontsf{Arial}
\usepackage[
    bibencoding=utf8,% кодировка bib файла
    sorting=none,% настройка сортировки списка литературы
    style=gost-numeric,% стиль цитирования и библиографии (по ГОСТ)
    language=autobib,% получение языка из babel/polyglossia, default: autobib % если ставить autocite или auto, то цитаты в тексте с указанием страницы, получат указание страницы на языке оригинала
    autolang=other,% многоязычная библиография
    clearlang=true,% внутренний сброс поля language, если он совпадает с языком из babel/polyglossia
    defernumbers=true,% нумерация проставляется после двух компиляций, зато позволяет выцеплять библиографию по ключевым словам и нумеровать не из большего списка
    sortcites=true,% сортировать номера затекстовых ссылок при цитировании (если в квадратных скобках несколько ссылок, то отображаться будут отсортированно, а не абы как)
    doi=false,% Показывать или нет ссылки на DOI
    isbn=false,% Показывать или нет ISBN
]{biblatex}
\DeclareSourcemap{ %модификация bib файла перед тем, как им займётся biblatex 
    \maps{
        \map{% перекидываем значения полей language в поля langid, которыми пользуется biblatex
            \step[fieldsource=language, fieldset=langid, origfieldval, final]
            \step[fieldset=language, null]
        }
    }
}
\emergencystretch=1em
\addbibresource{literature.bib}
%%%%%%%%%%%%%%%%%%%%%%%%
\author{\href{mailto:www-kirill.pilipenko@yandex.ru}{К.С.~Пилипенко} \href{https://github.com/PilipenkoKirill/MeasurementLabs}{\includegraphics[width=.5cm]{images/gitHubLogo.pdf}}} %Через \and можно добавить ещё авторов
\date{\the\year{}}

\title{Лабораторная работа №4 \\ \textit{Дискретные и непрерывные случайные величины. Распределение. Плотность распределения. Функции распределения}}
\begin{document}
\maketitle
\textbf{Дискретной случайной величиной} называется случайная величина, которая в результате испытания принимает отдельные значения с определёнными вероятностями.

\textbf{Непрерывной случайной величиной} называют случайную величину, которая в результате испытания принимает все значения из некоторого числового промежутка.

\textbf{Распределением} называют множество случайных величин \{$x_i$\} и соответствующее ему множество вероятностей \{$P(x_i)$\}.

\textbf{Функцией распределения} случайной величины $X$ называют функцию $F(x)$, значение которой в точке $x$ равно вероятности события \{$X \leqslant  x$\}.

\textbf{Плотность вероятности} или плотность распределения вероятностей случайной непрерывной величины x – это функция $\omega(x)$ удовлетворяющая условиям:  
\begin{equation}
    \omega(x) \geqslant 0, \int\limits^{+\infty}_{-\infty}\omega(x)dx = 1.
\end{equation}
Вероятность того, что величина Х заключена в интервале (a,b) при любых а < b равна:
\begin{equation}
    p(x) = \int\limits^b_a \omega(x)dx
\end{equation}
Функция распределения F(x) случайной величины X и плотность вероятности $\omega(x)$ связаны следующими соотношениями 
\begin{equation}
    F(x) = \int\limits^{x}_{-\infty}\omega(x)dx.
\end{equation}
\textbf{Распределение Пуассона} \\
Распределение Пуассона описывает дискретную случайную величину, представляющую собой число событий, произошедших за фиксированное время, при условии, что данные события происходят независимо друг от друга с некоторой фиксированной средней интенсивностью. 

Распределение Пуассона применимо, если:
\begin{enumerate}
    \item случайная величина принимает только положительные значения, 
    \item если длина интервала (например, t – время наблюдения) стремится к нулю, то вероятность одного события также стремится к нулю,
    \item события, относящиеся к неперекрывающимся интервалам, являются статистически независимыми.
\end{enumerate} 
Вероятность наблюдения n событий, произошедших за время t определяется формулой:
\begin{equation}
    P_n = \frac{{\bar{n}}^n}{n!}e^{-\bar{n}},
\end{equation}
$\bar{n}$ – математическое ожидание случайной величины (среднее количество событий за  промежуток времени t)

% \textbf{Пример.} Устройство состоит из 1000 элементов, работающих независимо один от другого. Вероятность отказа любого элемента в течение времени Т равна 0,002. Найти вероятность того, что за время Т откажут ровно три элемента.
% Решение\\
% Математическое ожидание в этом случае будет определяться как
% \begin{equation}
%     N_0 = \sum_{i=1}^{n} P_i,
% \end{equation} 
% где n~---~это число элементов, а $P_i$~---~вероятность выхода из строя одного элемента и тогда $N_0 = 2$. И тогда вероятность выхода из строя трёх элементов будет определяться следующим образом:
% \begin{equation}
%     P_3 = \frac{2^3}{3!}e^{-2} \approx 0,18 
% \end{equation}  
\textbf{Распределением Гаусса (нормальным распределением)} называют непрерывное распределение, имеющее следующую плотность вероятности:\\
\begin{equation} \label{normalDistribution}
    \omega(x)  = \frac{1}{\sigma \sqrt{2\pi}}e^{-\frac{(x-\bar{x})^2}{2\sigma^2}},
\end{equation}
где $\bar{x}$~---~среднее значение; $\sigma$~---~среднее квадратическое отклонение. Обоснование распределения Гаусса выходит за рамки нашего рассмотрения. Заметим, что нормальное или гауссово распределение может быть использовано при условии, что рассматриваемая случайная величина представляет собой выборку большого числа независимых случайных величин, максимальная из которых мала по сравнению с их суммой.

% \textbf{Биномиальное распределение (распределение Бернулли)} позволяет оценить количество происшедших  событий в серии из определенного числа независимых наблюдений (опытов), выполняемых в неизменных условиях:
% \begin{equation}
    
% \end{equation}
\textbf{Порядок построение графиков в Excel 2019 г.}
\begin{enumerate}
    \item Подготовьте два столбца значений условных $x$ и $y$;
    \item На вкладке \texttt{вставить} в области \texttt{диаграммы} выберете \texttt{вставить точечную или пузырьковую диаграмму} и из выпадающего списка выберете \texttt{точечная};
    \item Добавьте название осей нажав на \texttt{элементы диаграммы} (\begin{tikzpicture}
        \draw[green!60!black] (0,.15) -- (.3,.15);
        \draw[green!60!black] (.15,0) -- (.15,.3);
    \end{tikzpicture}) и поставив галочку возле соответствующего поля. После заполните появившиеся поля;
    \item Далее во вкладке \texttt{конструктор диаграмм} выберете \texttt{выбрать данные}. Добавьте новый ряд, указав название, столбцы $x$ и $y$;
    \item Остаётся только подтвердить выбранные параметры и ваш график готов!
\end{enumerate}

\textit{
\progress{}
\begin{enumerate}
    \item Составить график вероятности распределения Пуассона P(n) (не используя специальную для этого функцию) при $\bar{n} = 2$ на основе множества натуральных числе от 1 до 18. Как меняется распределение при изменении математического ожидания? Составьте ещё два ряда распределения в тех же осях при математическом ожидании равном 5 и 10;
    \item Составить графики плотности вероятности нормального распределения по формуле \ref{normalDistribution} при значениях $\bar{x}$ и $\sigma^2$ (функция принимает положительные и отрицательные непрерывные значения. Рекомендуется взять значения от -2 до 2 с шагом 0,1):
    \begin{itemize}
        \item $\bar{x}=0$, $\sigma^2 = 1$
        \item $\bar{x}=0$, $\sigma^2 = 5$
        \item $\bar{x}=2$, $\sigma^2 = 0,5$
    \end{itemize}
\end{enumerate}    
}
\questions{}
\begin{enumerate}
    % \item Что называют статистическим весом?
    \item Что такое математическое ожидание случайной величины? Какие свойства есть у математического ожидания?
    \item Как определяется дисперсия? Свойства дисперсии. Какой смысл дисперсии? 
    \item Приведите примеры использования распределения Пуассона.% в физике.
    \item Приведите примеры использования распределения Гаусса в физике.
    % \item Приведите примеры использования распределения Бернулли. 
\end{enumerate}
% \textbf{Методические рекомендации к заданию для обучающихся:} 

Выполнение практического задания проводится обучающимся
самостоятельно. Для расчетов используются результаты собственных
исследований полученных в ходе выполнения лабораторных работ по курсу <<механика>> (РЕКОМЕНДУЕТСЯ)
% , при отсутствии необходимого материала данные предоставляются преподавателем по запросу студента
. 
Работа выполняется в программе \href{https://mega.nz/folder/0EUyAKgC#4X5RubnYkoUUJYhwDDpKbg}{Microsoft Office Excel} или \href{https://www.libreoffice.org/download/download-libreoffice/?type=win-x86_64&version=7.4.0&lang=ru-RU}{LibreOffice Calc}. Результаты выполнения задания оформляются в файле с расширением ".xlsx". В названии файла укажите свою фамилию с инициалами, курс и группу.
\end{document}