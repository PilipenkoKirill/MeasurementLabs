%!TEX program = xelatex.exe
\documentclass[14pt,a4paper]{article}
% \usepackage[pdftex,
%     pdfauthor={К.С.~Пилипенко},
%     pdfsubject={The Subject},
%     pdfkeywords={Первое ключевое слово, второе ключевое слово},
%     pdfproducer={LuaLatex with hyperref},
%     pdfcreator={Lualatex},
%     % hidelinks
% ]{hyperref}
%%%%%%%%%%%%%%Пользовательские команды%%%%%%%%%
\usepackage[euler]{textgreek}
\usepackage{calc}
\usepackage{color}
\usepackage{listing}
\usepackage{svg}
\usepackage{hhline}
\usepackage{multirow}
\usepackage{latexsym,amsmath,amssymb,amsbsy,graphicx}
\usepackage{listings}
\usepackage{icomma}
\usepackage[obeyspaces]{url} %Позволяет прописывать путь к файлам
\usepackage[american,siunitx]{circuitikz}
\def\centerarc[#1][#2](#3:#4:#5){\draw[#1]($(#2)+({#5*cos(#3)},{#5*sin(#3)})$)}
\usepackage[version=4]{mhchem} % the canonical chemistry package (example: \ce{^{32}_{15}P})
\usepackage{graphicx}
\usepackage{pgfplots}
\usepackage{wrapfig}
\graphicspath{{images/}}
\DeclareGraphicsExtensions{.pdf,.png,.jpg}
\usepackage{hyperref}
\hypersetup{
    colorlinks=true,
    linkcolor=blue,
    filecolor=magenta,      
    urlcolor=cyan,
    pdfauthor={К.С.~Пилипенко},
    pdfsubject={Основы обработки результатов исследования в физике},
    % pdfkeywords={Первое ключевое слово, второе ключевое слово},
    pdfproducer={LuaLatex with hyperref},
    pdfcreator={Lualatex},
    pdfpagemode=FullScreen,
    }
\urlstyle{same}
%Форматирование разделов
\usepackage{titlesec}
\usepackage{multicol} %Многоколоночный список
%%% Кастомизация разделов
\titleformat
{\section} % command
[block] % shape
{\normalfont\bfseries\itshape} % format
{Задание №\thesection.}{0.5em}{} % label
\newcommand{\progress}{\begin{center}\large\bfseries Ход работы\end{center}}
\newcommand{\questions}{\begin{center}\large\bfseries  Контрольные вопросы \end{center}}
%%%%%%%%%%%%%%%%%%%%%%%%Оформление по ГОСТУ
\usepackage{fontspec}
\setmainfont[Renderer=Basic,Ligatures={TeX}]{Times New Roman}
% \usepackage[english,russian]{babel} %Поддержка русской локализации
\setmonofont{dejavusansmono} % verbatim с кириллицей
\usepackage[14pt]{extsizes} % для того чтобы задать нестандартный 14-ый размер шрифта
\usepackage{indentfirst} %Задаёт отступ самого первого абзаца
\setlength\parindent{1.25cm}
\usepackage[a4paper, left=3cm, top=1.5cm, right=1.5cm, bottom=2cm]{geometry}
\usepackage{setspace}
\sloppy %Выравнивание текст по ширине и решение проблемы переполнением строки
\onehalfspacing %Полуторный интервал
%%%%%%%%%%%%%%%%%%%%%%%%%%%%% Добавление списка литературы
\usepackage{polyglossia}[2014/05/21]            % Поддержка многоязычности (fontspec подгружается автоматически)
\setmainlanguage[babelshorthands=true]{russian}  % Язык по-умолчанию русский с поддержкой приятных команд пакета babel
\setotherlanguage{english}                       % Дополнительный язык = английский (в американской вариации по-умолчанию)
\setmonofont{Courier New}
\newfontfamily\cyrillicfonttt{Courier New}
    \defaultfontfeatures{Ligatures=TeX,Mapping=tex-text}
\setmainfont{Times New Roman}
\newfontfamily\cyrillicfont{Times New Roman}
\setsansfont{Arial}
\newfontfamily\cyrillicfontsf{Arial}
\usepackage[
    bibencoding=utf8,% кодировка bib файла
    sorting=none,% настройка сортировки списка литературы
    style=gost-numeric,% стиль цитирования и библиографии (по ГОСТ)
    language=autobib,% получение языка из babel/polyglossia, default: autobib % если ставить autocite или auto, то цитаты в тексте с указанием страницы, получат указание страницы на языке оригинала
    autolang=other,% многоязычная библиография
    clearlang=true,% внутренний сброс поля language, если он совпадает с языком из babel/polyglossia
    defernumbers=true,% нумерация проставляется после двух компиляций, зато позволяет выцеплять библиографию по ключевым словам и нумеровать не из большего списка
    sortcites=true,% сортировать номера затекстовых ссылок при цитировании (если в квадратных скобках несколько ссылок, то отображаться будут отсортированно, а не абы как)
    doi=false,% Показывать или нет ссылки на DOI
    isbn=false,% Показывать или нет ISBN
]{biblatex}
\DeclareSourcemap{ %модификация bib файла перед тем, как им займётся biblatex 
    \maps{
        \map{% перекидываем значения полей language в поля langid, которыми пользуется biblatex
            \step[fieldsource=language, fieldset=langid, origfieldval, final]
            \step[fieldset=language, null]
        }
    }
}
\emergencystretch=1em
\addbibresource{literature.bib}
%%%%%%%%%%%%%%%%%%%%%%%%
\author{\href{mailto:www-kirill.pilipenko@yandex.ru}{К.С.~Пилипенко} \href{https://github.com/PilipenkoKirill/MeasurementLabs}{\includegraphics[width=.5cm]{images/gitHubLogo.pdf}}} %Через \and можно добавить ещё авторов
\date{\the\year{}}

\usepackage{pgfplotstable}
\usepackage{booktabs}
\pgfplotstableset{
every head row/.style={before row=\toprule,after row=\midrule},
every last row/.style={after row=\bottomrule}}
\title{Лабораторная работа №8 \\ \textit{Введение в Mathcad}}
\begin{document}
\maketitle

\progress{}
\section{Установка программы}
\begin{enumerate}
	\item Скачайте программный пакет по \href{https://mega.nz/folder/BJVHALjD#w-peT3ym3XjbBn78NVb9OA}{ссылке};
	\item Зайдите в папку «PTC.LICENSE.WINDOWS.2022-04-21-SSQ» по пути \path{_SolidSQUAD_\_SolidSQUAD_\PTC.LICENSE.WINDOWS.2022-04-21-SSQ}
    и запустите \texttt{FillLicense.bat}. Должен появится новый файл с расширением \path{.dat};
	\item Зайдите по пути \path{C:\Program Files} и создайте там пустую папку \path{PTC}, при необходимости предоставьте разрешения администратора. Сгенерированный файл \path{PTC_D_SSQ.dat} скопируйте в \path{C:\Program Files\PTC};
	\item Создайте переменную среды:
	\begin{enumerate}
		% \item Нажмите \texttt{Пуск} и в поисковике наберите \texttt{Переменные среды}
		\item Нажмите клавиши \texttt{Win+R} на клавиатуре, введите \path{sysdm.cpl}, а затем \texttt{Enter};
		\item На вкладке \texttt{Дополнительно} нажмите кнопку \texttt{Переменные среды…};
		\item В разделе \texttt{Переменные среды пользователя} нажмите кнопку \texttt{Создать}. Укажите имя переменной \path{PTC_D_LICENSE_FILE} и путь к исполняемому файлу \path{C:\Program Files\PTC\PTC_D_SSQ.dat}, затем подтвердить изменения.
	\end{enumerate}
	\item Установите PTC Mathcad Prime 9.0.0.0 Win64. Для этого необходимо иметь предустановленную на компьютере программу DAEMON Tools или её свободный аналог --- \href{https://mega.nz/file/Fd0T1JDA#HEkbzw68-mowukA16ODSRbXf7mGBfPjmXqBnG31CxF8}{Virtual CloneDrive};
	\item Все файлы в папке \path{_SolidSQUAD_\Mathcad Prime 9.0.0.0} скопировать и вставить с заменой в папку с установленной программой (по умолчанию: \path{C:\Program Files\PTC\Mathcad Prime 9.0.0.0});
	\item При первом запуске Mathcad Prime 9.0 при запросе лицензии в PTC Mathcad License Wizard: Выберите «Настроить продукт для использования существующей лицензии» > «Далее» > «Файл» > «Обзор» > (перейдите к файлу \path{PTC_D_SSQ.dat}, сохраненному на вашем компьютере) > Настройка лицензии > Выход.
\end{enumerate}
\section{Арифметические вычисления}
\noindent Вычислить значение выражения
\begin{enumerate}
	\item $\left(\frac{(2,7-0,8 \cdot 1,6) \cdot \frac{1}{3}}{(5,2-1,4 \cdot 2,7): \frac{3}{7}}+0,125\right): 7,1+0,38$;
	\item $\frac{\left(\frac{17}{40}+0,6-0,005\right) \cdot 1,7}{\frac{5}{6}+\frac{1}{3}-\frac{23}{30}}+\frac{4,75+\frac{1}{2}}{31: \frac{5}{7}}$.
\end{enumerate}
Задать ранжированную переменную:
\begin{enumerate}
	\item переменная х меняется в пределах от – 5 до 5 с шагом 1;
	\item переменная х меняется в пределах от 1 до 2 с шагом 0,1.
\end{enumerate}
\section{Определение функции и пределы}
\noindent Задать функцию и определить ее значения в указанной области:

\begin{enumerate}
	\item $\mathrm{y}(\mathrm{x})=\sqrt{x^3+2 x^2-1}+\frac{x^3+2}{x-1},\quad \mathrm{x}=2 . .10$;
	\item $\Phi(\varphi)=\operatorname{tg}\left(\sin ^2 \varphi\right)-\frac{\cos \varphi}{1+\sin \varphi}, \quad \varphi=0, \pi / 4 . . \pi$.
\end{enumerate}
\noindent Вычислите пределы:
\begin{multicols}{2}
\begin{enumerate}
	\item $\lim\limits_{x \rightarrow 1} \frac{x^3-1}{\ln x}$;
	\item $\lim\limits_{x \rightarrow 0}\left(\frac{1}{x}-\frac{1}{e^x-1}\right)$.
\end{enumerate}
\end{multicols}
\section{Дифференцирование и интегрирование}
\noindent Вычислите символьно и численно (значения переменной задайте самостоятельно):
\begin{multicols}{2}
\begin{enumerate}
	\item $\frac{d}{d z}\left(z^3-\operatorname{tg}(z) /\left(z^2-2\right)\right)$;
	\item $\frac{d^2}{d y^2}\left(a \sin (y)+\ln \left(y^3\right)^2\right)$.
\end{enumerate}
\end{multicols}
\noindent Найти неопределенные интегралы:
\begin{multicols}{2}
\begin{enumerate}
	\item $\int \frac{\sqrt{x}-2 \sqrt[3]{x^2}+2}{\sqrt[4]{x}} dx$;
	\item $\int \frac{\ln x d x}{x \sqrt{1+\ln x}}$.
\end{enumerate}
\end{multicols}
\noindent Найти определенные интегралы:
\begin{multicols}{2}
\begin{enumerate}
	\item $\int\limits_{-\pi/4}^{2\pi} \operatorname{arctg}(\sqrt[3]{6 x-1}) d x$;
	\item $\int\limits_{0}^{\infty} x^4e^{-5x^2}dx$. 
\end{enumerate}
\end{multicols}
% \questions{}
% \begin{enumerate}
% 	\item 
% \end{enumerate}
\nocite{Plis2003,Krestlev2010,MathCADshortcuts,Ochkov2016}
\printbibliography[title={Рекомендуемая литература}]
\end{document}