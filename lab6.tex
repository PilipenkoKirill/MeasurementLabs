%!TEX program = xelatex.exe
\documentclass[14pt,a4paper]{article}
% \usepackage[pdftex,
%     pdfauthor={К.С.~Пилипенко},
%     pdfsubject={The Subject},
%     pdfkeywords={Первое ключевое слово, второе ключевое слово},
%     pdfproducer={LuaLatex with hyperref},
%     pdfcreator={Lualatex},
%     % hidelinks
% ]{hyperref}
%%%%%%%%%%%%%%Пользовательские команды%%%%%%%%%
\usepackage[euler]{textgreek}
\usepackage{calc}
\usepackage{color}
\usepackage{listing}
\usepackage{svg}
\usepackage{hhline}
\usepackage{multirow}
\usepackage{latexsym,amsmath,amssymb,amsbsy,graphicx}
\usepackage{listings}
\usepackage{icomma}
\usepackage[obeyspaces]{url} %Позволяет прописывать путь к файлам
\usepackage[american,siunitx]{circuitikz}
\def\centerarc[#1][#2](#3:#4:#5){\draw[#1]($(#2)+({#5*cos(#3)},{#5*sin(#3)})$)}
\usepackage[version=4]{mhchem} % the canonical chemistry package (example: \ce{^{32}_{15}P})
\usepackage{graphicx}
\usepackage{pgfplots}
\usepackage{wrapfig}
\graphicspath{{images/}}
\DeclareGraphicsExtensions{.pdf,.png,.jpg}
\usepackage{hyperref}
\hypersetup{
    colorlinks=true,
    linkcolor=blue,
    filecolor=magenta,      
    urlcolor=cyan,
    pdfauthor={К.С.~Пилипенко},
    pdfsubject={Основы обработки результатов исследования в физике},
    % pdfkeywords={Первое ключевое слово, второе ключевое слово},
    pdfproducer={LuaLatex with hyperref},
    pdfcreator={Lualatex},
    pdfpagemode=FullScreen,
    }
\urlstyle{same}
%Форматирование разделов
\usepackage{titlesec}
\usepackage{multicol} %Многоколоночный список
%%% Кастомизация разделов
\titleformat
{\section} % command
[block] % shape
{\normalfont\bfseries\itshape} % format
{Задание №\thesection.}{0.5em}{} % label
\newcommand{\progress}{\begin{center}\large\bfseries Ход работы\end{center}}
\newcommand{\questions}{\begin{center}\large\bfseries  Контрольные вопросы \end{center}}
%%%%%%%%%%%%%%%%%%%%%%%%Оформление по ГОСТУ
\usepackage{fontspec}
\setmainfont[Renderer=Basic,Ligatures={TeX}]{Times New Roman}
% \usepackage[english,russian]{babel} %Поддержка русской локализации
\setmonofont{dejavusansmono} % verbatim с кириллицей
\usepackage[14pt]{extsizes} % для того чтобы задать нестандартный 14-ый размер шрифта
\usepackage{indentfirst} %Задаёт отступ самого первого абзаца
\setlength\parindent{1.25cm}
\usepackage[a4paper, left=3cm, top=1.5cm, right=1.5cm, bottom=2cm]{geometry}
\usepackage{setspace}
\sloppy %Выравнивание текст по ширине и решение проблемы переполнением строки
\onehalfspacing %Полуторный интервал
%%%%%%%%%%%%%%%%%%%%%%%%%%%%% Добавление списка литературы
\usepackage{polyglossia}[2014/05/21]            % Поддержка многоязычности (fontspec подгружается автоматически)
\setmainlanguage[babelshorthands=true]{russian}  % Язык по-умолчанию русский с поддержкой приятных команд пакета babel
\setotherlanguage{english}                       % Дополнительный язык = английский (в американской вариации по-умолчанию)
\setmonofont{Courier New}
\newfontfamily\cyrillicfonttt{Courier New}
    \defaultfontfeatures{Ligatures=TeX,Mapping=tex-text}
\setmainfont{Times New Roman}
\newfontfamily\cyrillicfont{Times New Roman}
\setsansfont{Arial}
\newfontfamily\cyrillicfontsf{Arial}
\usepackage[
    bibencoding=utf8,% кодировка bib файла
    sorting=none,% настройка сортировки списка литературы
    style=gost-numeric,% стиль цитирования и библиографии (по ГОСТ)
    language=autobib,% получение языка из babel/polyglossia, default: autobib % если ставить autocite или auto, то цитаты в тексте с указанием страницы, получат указание страницы на языке оригинала
    autolang=other,% многоязычная библиография
    clearlang=true,% внутренний сброс поля language, если он совпадает с языком из babel/polyglossia
    defernumbers=true,% нумерация проставляется после двух компиляций, зато позволяет выцеплять библиографию по ключевым словам и нумеровать не из большего списка
    sortcites=true,% сортировать номера затекстовых ссылок при цитировании (если в квадратных скобках несколько ссылок, то отображаться будут отсортированно, а не абы как)
    doi=false,% Показывать или нет ссылки на DOI
    isbn=false,% Показывать или нет ISBN
]{biblatex}
\DeclareSourcemap{ %модификация bib файла перед тем, как им займётся biblatex 
    \maps{
        \map{% перекидываем значения полей language в поля langid, которыми пользуется biblatex
            \step[fieldsource=language, fieldset=langid, origfieldval, final]
            \step[fieldset=language, null]
        }
    }
}
\emergencystretch=1em
\addbibresource{literature.bib}
%%%%%%%%%%%%%%%%%%%%%%%%
\author{\href{mailto:www-kirill.pilipenko@yandex.ru}{К.С.~Пилипенко} \href{https://github.com/PilipenkoKirill/MeasurementLabs}{\includegraphics[width=.5cm]{images/gitHubLogo.pdf}}} %Через \and можно добавить ещё авторов
\date{\the\year{}}

\title{Лабораторная работа №6 \\ \textit{Использование программы Excel для решения задач на законы Кирхгофа.}}
\begin{document}
\maketitle
Системой линейных уравнений с \emph{n} неизвестными называют систему
уравнений вида
\begin{equation} \label{sysLinEquat}
    \begin{cases}
        a_{11}x_1 + a_{12}x_2 + \ldots + a_{1n}x_n = b_1 \\
        a_{21}x_1 + a_{22}x_2 + \ldots + a_{2n}x_n = b_2 \\
        \vdots \\
        a_{n1}x_1 + a_{n2}x_2 + \ldots + a_{nn}x_n = b_n 
    \end{cases}
\end{equation}
где \emph{x\textsubscript{1},
x\textsubscript{2},\ldots,x\textsubscript{n} --} неизвестные величины,
\emph{a\textsubscript{nm}} (\emph{n=1,2,\ldots,n, m=1, 2,\ldots,n}) --
числовые коэффициенты (коэффициенты при неизвестных),
\emph{b\textsubscript{1}, b\textsubscript{2},\ldots,b\textsubscript{n}}
-- свободные члены. Форма записи алгебраической линейной системы в виде
\ref{sysLinEquat} называется нормальной, целое число \emph{n} -- размерность системы.
Решением системы уравнений \ref{sysLinEquat} является набор значений
\emph{x\textsubscript{1}, x\textsubscript{2},\ldots,x\textsubscript{n}},
который все уравнения системы обращает в тождества. Система уравнений
называется совместной, если она имеет хотя бы одно решение, и
несовместной \emph{--} в противном случае. Совместная система называется
определенной\emph{,} если она имеет единственное решение, и
неопределенной\emph{,} если она имеет больше одного решения.

Формулы для нахождения решений системы по методу Крамера имеют вид
дробей: 
% \footnote{Линейная алгебраическая система несовместна (не имеет решений), если ∆=det \emph{A=0.}
\[ x_1 = \frac{|A_1|}{|A|}, \quad  x_2 =\frac{|A_2|}{|A|}, \quad \ldots, \quad x_n = \frac{|A_n|}{|A|}\]
где $|A|, |A_1|, |A_2|, \ldots, |A_n|$~---~определители матрицы, составленные по системе уравнений \ref{sysLinEquat}
\begin{equation}
    \begin{aligned}
        |A|=
        \begin{vmatrix}
            a_{11} & a_{12}  & \cdots   & a_{1n}   \\
            a_{21} & a_{22}  & \cdots   & a_{2n}  \\
            \vdots & \vdots  & \ddots   & \vdots  \\
            a_{n1} & a_{n2}  & \cdots\  & a_{nn}  \\
        \end{vmatrix}, \;
    |A_1|=
    \begin{vmatrix}
        b_{1} & a_{12}  & \cdots   & a_{1n}   \\
        b_{2} & a_{22}  & \cdots   & a_{2n}  \\
        \vdots & \vdots  & \ddots   & \vdots  \\
        b_{n} & a_{n2}  & \cdots\  & a_{nn}  \\
    \end{vmatrix}, \\
    |A_2|=
    \begin{vmatrix}
        a_{11} & b_{1}  & \cdots   & a_{1n}   \\
        a_{21} & b_{2}  & \cdots   & a_{2n}  \\
        \vdots & \vdots  & \ddots   & \vdots  \\
        a_{n1} & b_{n}  & \cdots\  & a_{nn}  \\
    \end{vmatrix}, \ldots ,  
    |A_n|=
    \begin{vmatrix}
        a_{11} & a_{12}  & \cdots   & b_{n}   \\
        a_{21} & a_{22}  & \cdots   & b_{n}  \\
        \vdots & \vdots  & \ddots   & \vdots  \\
        a_{n1} & a_{n2}  & \cdots\  & b_{n}  \\
    \end{vmatrix}
\end{aligned}
\end{equation}
\begin{center}
    \textbf{Способ решения системы линейных уравнений в Excel}
\end{center}
Из системы уравнений \ref{sysLinEquat} и правила перемножения матриц можно составить выражение:
\begin{equation}
    \begin{pmatrix}
        a_{11} & a_{12}  & \cdots   & a_{1n}   \\
        a_{21} & a_{22}  & \cdots   & a_{2n}  \\
        \vdots & \vdots  & \ddots   & \vdots  \\
        a_{n1} & a_{n2}  & \cdots\  & a_{nn}  \\
    \end{pmatrix} \times
    \begin{pmatrix}
        x_{1}   \\
        x_{2}  \\
        \vdots  \\
        x_{n}   \\
    \end{pmatrix} = 
    \begin{pmatrix}
        b_{n}   \\
        b_{n}  \\
        \vdots  \\
        b_{n}  \\
    \end{pmatrix}\; \text{или} \; A \times X = B,
\end{equation}
и тогда будет справедливо уравнение 
\begin{equation} \label{Xmatrix}
    X = A^{-1}\times B,
\end{equation}
где $A^{-1}$~---~обратная матрица матрицы $A$. % (в Excel используется функция  \verb=МОБР=).
Для перемножения матриц используется функция \verb=МУМНОЖ= (проще всего начать вводить функцию и нажать на \verb=TAB=).

Другими словами, чтобы найти матрицу, состоящую из неизвестных величин, нужно найти обратную матрицу матрицы, состоящей из числовых коэффициентов, а потом перемножить её с матрицей из свободных членов. Помните, что перемножение матриц \emph{некоммутативно}!

\begin{center}
    \textbf{ Правила Кирхгофа}
\end{center}
\begin{wrapfigure}{r}{.25\textwidth}
    \begin{circuitikz}[american]
        \draw (2,0) to[short,i=$I_3$, -*] (2,2) to[short, i=$I_2$] (4,4);
        \draw (0,4) to[short,i=$I_1$] (2,2);
    \end{circuitikz}
\end{wrapfigure}

\textbf{Первое правило Кирхгофа (правило токов Кирхгофа):} алгебраическая сумма токов ветвей, сходящихся в каждом узле любой цепи, равна нулю. При этом направленный к узлу ток принято считать положительным, а направленный от узла — отрицательным:
\begin{equation}
    \sum\limits_j^n I_j = 0
\end{equation}

\begin{wrapfigure}{r}{.3\textwidth}
    \begin{circuitikz}[american]
        \draw(2,1) to[generic, l=$R_1$, i<=$I_1$] (0.6,3.1) to[battery1, v=$\varepsilon_1$, -*] (0,4) -- (-0.5,4.5);
        \draw(0,4) to[generic, l=$R_3$, i<=$I_3$] (2.6,4) to[battery1, v=$\varepsilon_3$, -*] (4,4) -- (4.5,4.5);
        \draw(4,4) to[generic, l=$R_2$, i<=$I_2$] (2.6,1.9) to[battery1, v=$\varepsilon_2$, -*] (2,1) -- (2,0.5);
        \draw[thick, ->, red] (2.6,2.85) arc (0:320:0.6cm);
    \end{circuitikz}
\end{wrapfigure}
\textbf{Второе правило Кирхгофа (правило напряжений Кирхгофа):} алгебраическая сумма напряжений на резистивных элементах замкнутого контура равна алгебраической сумме ЭДС, входящих в этот контур.

\begin{equation}
    \sum\limits_k^n\varepsilon_k = \sum\limits_k^m I_kR_k
\end{equation}

\textbf{Пример}

\begin{tabular}{p{3cm}|l}
    Дано & \multirow{7}{*}{\begin{circuitikz}[american]
        \draw(0,0) to[battery1, v_<=$\varepsilon_1$, invert](3,0) to[generic, l_=$R_2$, i_<=$I_2$, invert] (6,0) to (6,3) to[battery1, v_=$\varepsilon_2$] (3,3) to[generic,l_=$R_1$,i=$I_1$] (0,3) to (0,0);
        \draw(3,0) to[generic,l=$R_3$,i=$I_3$,*-*](3,3);
        \draw[thick, ->, red, xshift=0.6cm] (1.5,1.5) arc (0:320:0.6cm) node[pos=0, xshift=-0.6cm] {I};
        \draw[thick, ->, red, anchor=center, xshift=0.6cm] (4.5,1.5) arc (0:-320:0.6cm) node[pos=0, xshift=-0.6cm] {II};
    \end{circuitikz}}\\
    $R_1 = 100$~Ом \\ 
    $R_2 = 150$~Ом \\
    $R_3 = 150$~Ом \\
    $E_1 = 75$~В\\
    $E_2 = 100$~В\\
    \hhline{-~}
    $I_1,I_2,I_3 - ?$
\end{tabular}



Используя \textbf{первое правило Кирхгофа}, можно записать n-1 уравнений для цепи, где n~---~количество узлов. В нашем случае количество узлов n=2, а значит нужно составить только одно уравнение:
\begin{equation} \label{firstLow}
    I_3 -I_1 - I_2 = 0
\end{equation}

Затем используя \textbf{второе правило} составим уравнения для первого и второго контуров цепи. Направления обхода выбраны произвольными, при этом если направление тока через резистор совпадает с направлением обхода, берем со знаком плюс, и наоборот если не совпадает, то со знаком минус. Аналогично с источниками ЭДС.

На примере первого контура --- ток $I_1$ и $I_3$ совпадают с направлением обхода контура (против часовой стрелки), ЭДС $\varepsilon_1$ также совпадает, поэтому берем их со знаком плюс.

Уравнения для первого и второго контуров по второму закону будут:
\begin{equation} \label{secondLow}
    \begin{split}
        R_1I_1 + R_3I_3 & = \varepsilon_1 \\
        R_2I_2 +R_3I_3 & = \varepsilon_2
    \end{split}
\end{equation}

Уравнения \ref{firstLow} и \ref{secondLow} образуют систему

\begin{equation}
    \begin{cases}
        I_3 -I_1 - I_2 = 0\\
        R_1I_1 + R_3I_3 = \varepsilon_1 \\
        R_2I_2 +R_3I_3 = \varepsilon_2,
    \end{cases}
\end{equation}
Подставив значения из <<Дано>> получим:
\begin{equation}
    \begin{cases}
        -I_1 -I_2 + I_3 = 0\\
        100I_1 + 150I_3 = 75 \\
        150I_2 + 150I_3 = 100,
    \end{cases}
\end{equation}
И тогда матрицы из коэффициентов и свободных членов соответственно:
\begin{equation}
    A = \begin{pmatrix}
        -1 & -1   & 1   \\
        100 & 0   & 150  \\
        0 & 150   & 150  \\
    \end{pmatrix}; \quad B =
    \begin{pmatrix}
        0  \\
        75  \\
        100 \\
    \end{pmatrix}
\end{equation}
\progress{}


Используя первое и второе правила Кирхгофа получите систему уравнений для каждой электрической схемы. Используя уравнение \ref{Xmatrix} получите матрицу токов, проходящих через каждый элемент цепи.
\section{}
\begin{wrapfigure}[4]{l}{.3\textwidth}
        \begin{circuitikz}[american, scale=.53, transform shape]
            \draw (0,0) to [battery1, v_=$\varepsilon_1$, l=$r_1$] (0,-3);
            \draw (2.3,0) to [battery1, v_=$\varepsilon_2$, l=$r_2$, *-*] (2.3,-3);
            \draw(0,0) to (2,0) to [generic,l_=$R_{1}$,i=$I_{1}$] (6,0) to (8,0);
            \draw(6,0) to[generic,l_=$R_{2}$,i=$I_{2}$,*-*](6,-3);
            \draw (6,0) to (8,0) to [I, l_=$G$,i=$I_{G}$] (8,-3);
            \draw(8,-3) to [generic,l_=$R_{3}$] (0,-3);
        \end{circuitikz}
\end{wrapfigure}
Найти величину тока через гальванометр $G$, включенный в цепь, если $\varepsilon_1=\varepsilon_2=1,5$ В, внутренние сопротивления источников ЭДС равны $r_1=r_2=0,5$ Ом, $R_1=R_2=2$ Ом, $R_3=1$ Ом, $R_G=3$ Ом.

\section{}
\begin{wrapfigure}[4]{l}{.3\textwidth}
    \begin{circuitikz}[american, scale=.65, transform shape]
        \draw(0,0) to[generic, l=$R_1$] (0,2) to (1,2) to[generic, l=$R_2$] (1,0) to (0,0);
        \draw(0.5,2) to[battery1, v_<=$\varepsilon_1$,*-, invert] (0.5,3) to (5.5,3) to [short,-*] (5.5,2);
        \draw(5,0) to[generic, l=$R_4$] (5,2) to (6,2) to[generic, l=$R_5$] (6,0) to (5,0);
        \draw(5.5,0) to[short, *-] (5.5,-0.5) to[generic, l=$R_6$] (3,-0.5) to[generic, l=$R_7$] (0.5,-0.5) to [short, -*] (0.5,0);
        \draw(3,-0.5) to[generic, l=$R_3$, *-] (3,2) to[battery1, v_<=$\varepsilon_2$,-*, invert] (3,3);
\end{circuitikz}
\end{wrapfigure}
Найти величины токов во всех участках цепи, если $\varepsilon_1=20$ В, $\varepsilon_2=33$ В, $r_1=0,2$ Ом, $r_2=0,5$ Ом, $R_1=0,8$ Ом, $R_2=2$ Ом, $R_3=14,3$ Ом, $R_4=23,4$ Ом, $R_5=34,3$ Ом, $R_6=40,3$ Ом, $R_7=12$ Ом.

\section{}
\begin{wrapfigure}[7]{l}{.3\textwidth}
    \begin{circuitikz}[american, scale=.7, transform shape]
        \draw(0,0) to [battery1, v<=$\varepsilon_1$, l_=$r_1$, invert]++ (0,1.5) to[generic, l=$R_1$] (0,4) to (2.2,4) to[generic, l=$R_4$] (5,4) to[generic, l=$R_3$] (5,1.5) to[battery1, v_=$\varepsilon_3$, l=$r_3$] (5,0) to (0,0);
        \draw(2.2,0) to[battery1, v<=$\varepsilon_2$, l_=$r_2$,*-,invert] (2.2,1.5) to[generic, l=$R_2$, -*] (2.2,4);
    \end{circuitikz}
\end{wrapfigure}
Определить величину токов во всех участках цепи и мощность, развиваемую источниками ЭДС, если $\varepsilon_1=6$ В, $\varepsilon_2=10$ В, $E_3=20$ В, $r_1=0,2$ Ом, $r_2=0,2$ Ом, $r_3=0,4$ Ом, $R_1=19,8$ Ом, $R_2=45,8$ Ом, $R_3=100$ Ом, $R_4=99,6$ Ом. 
% \begin{enumerate}
%     \item 
% \end{enumerate} 
\end{document}