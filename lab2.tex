%!TEX program = xelatex.exe
\documentclass[14pt,a4paper]{article}
% \usepackage[pdftex,
%     pdfauthor={К.С.~Пилипенко},
%     pdfsubject={The Subject},
%     pdfkeywords={Первое ключевое слово, второе ключевое слово},
%     pdfproducer={LuaLatex with hyperref},
%     pdfcreator={Lualatex},
%     % hidelinks
% ]{hyperref}
%%%%%%%%%%%%%%Пользовательские команды%%%%%%%%%
\usepackage[euler]{textgreek}
\usepackage{calc}
\usepackage{color}
\usepackage{listing}
\usepackage{svg}
\usepackage{hhline}
\usepackage{multirow}
\usepackage{latexsym,amsmath,amssymb,amsbsy,graphicx}
\usepackage{listings}
\usepackage{icomma}
\usepackage[obeyspaces]{url} %Позволяет прописывать путь к файлам
\usepackage[american,siunitx]{circuitikz}
\def\centerarc[#1][#2](#3:#4:#5){\draw[#1]($(#2)+({#5*cos(#3)},{#5*sin(#3)})$)}
\usepackage[version=4]{mhchem} % the canonical chemistry package (example: \ce{^{32}_{15}P})
\usepackage{graphicx}
\usepackage{pgfplots}
\usepackage{wrapfig}
\graphicspath{{images/}}
\DeclareGraphicsExtensions{.pdf,.png,.jpg}
\usepackage{hyperref}
\hypersetup{
    colorlinks=true,
    linkcolor=blue,
    filecolor=magenta,      
    urlcolor=cyan,
    pdfauthor={К.С.~Пилипенко},
    pdfsubject={Основы обработки результатов исследования в физике},
    % pdfkeywords={Первое ключевое слово, второе ключевое слово},
    pdfproducer={LuaLatex with hyperref},
    pdfcreator={Lualatex},
    pdfpagemode=FullScreen,
    }
\urlstyle{same}
%Форматирование разделов
\usepackage{titlesec}
\usepackage{multicol} %Многоколоночный список
%%% Кастомизация разделов
\titleformat
{\section} % command
[block] % shape
{\normalfont\bfseries\itshape} % format
{Задание №\thesection.}{0.5em}{} % label
\newcommand{\progress}{\begin{center}\large\bfseries Ход работы\end{center}}
\newcommand{\questions}{\begin{center}\large\bfseries  Контрольные вопросы \end{center}}
%%%%%%%%%%%%%%%%%%%%%%%%Оформление по ГОСТУ
\usepackage{fontspec}
\setmainfont[Renderer=Basic,Ligatures={TeX}]{Times New Roman}
% \usepackage[english,russian]{babel} %Поддержка русской локализации
\setmonofont{dejavusansmono} % verbatim с кириллицей
\usepackage[14pt]{extsizes} % для того чтобы задать нестандартный 14-ый размер шрифта
\usepackage{indentfirst} %Задаёт отступ самого первого абзаца
\setlength\parindent{1.25cm}
\usepackage[a4paper, left=3cm, top=1.5cm, right=1.5cm, bottom=2cm]{geometry}
\usepackage{setspace}
\sloppy %Выравнивание текст по ширине и решение проблемы переполнением строки
\onehalfspacing %Полуторный интервал
%%%%%%%%%%%%%%%%%%%%%%%%%%%%% Добавление списка литературы
\usepackage{polyglossia}[2014/05/21]            % Поддержка многоязычности (fontspec подгружается автоматически)
\setmainlanguage[babelshorthands=true]{russian}  % Язык по-умолчанию русский с поддержкой приятных команд пакета babel
\setotherlanguage{english}                       % Дополнительный язык = английский (в американской вариации по-умолчанию)
\setmonofont{Courier New}
\newfontfamily\cyrillicfonttt{Courier New}
    \defaultfontfeatures{Ligatures=TeX,Mapping=tex-text}
\setmainfont{Times New Roman}
\newfontfamily\cyrillicfont{Times New Roman}
\setsansfont{Arial}
\newfontfamily\cyrillicfontsf{Arial}
\usepackage[
    bibencoding=utf8,% кодировка bib файла
    sorting=none,% настройка сортировки списка литературы
    style=gost-numeric,% стиль цитирования и библиографии (по ГОСТ)
    language=autobib,% получение языка из babel/polyglossia, default: autobib % если ставить autocite или auto, то цитаты в тексте с указанием страницы, получат указание страницы на языке оригинала
    autolang=other,% многоязычная библиография
    clearlang=true,% внутренний сброс поля language, если он совпадает с языком из babel/polyglossia
    defernumbers=true,% нумерация проставляется после двух компиляций, зато позволяет выцеплять библиографию по ключевым словам и нумеровать не из большего списка
    sortcites=true,% сортировать номера затекстовых ссылок при цитировании (если в квадратных скобках несколько ссылок, то отображаться будут отсортированно, а не абы как)
    doi=false,% Показывать или нет ссылки на DOI
    isbn=false,% Показывать или нет ISBN
]{biblatex}
\DeclareSourcemap{ %модификация bib файла перед тем, как им займётся biblatex 
    \maps{
        \map{% перекидываем значения полей language в поля langid, которыми пользуется biblatex
            \step[fieldsource=language, fieldset=langid, origfieldval, final]
            \step[fieldset=language, null]
        }
    }
}
\emergencystretch=1em
\addbibresource{literature.bib}
%%%%%%%%%%%%%%%%%%%%%%%%
\author{\href{mailto:www-kirill.pilipenko@yandex.ru}{К.С.~Пилипенко} \href{https://github.com/PilipenkoKirill/MeasurementLabs}{\includegraphics[width=.5cm]{images/gitHubLogo.pdf}}} %Через \and можно добавить ещё авторов
\date{\the\year{}}

\title{Лабораторная работа №2 \\ \textit{Погрешность косвенных измерений}}
\begin{document}
\maketitle
Пусть некоторая величина $f$ зависит от $n$ величин, получаемых в результате прямого измерения  $x_1, x_2, \ldots, x_n$ (это могут быть температура, напряжение, длина и др.), причём вид этой зависимости $f = f(x_1, x_2, \ldots, x_n)$ известен и называется \textbf{рабочей формулой}, тогда, используя выражение для полного дифференциала функции нескольких аргументов, абсолютная погрешность косвенных измерений будет определятся по формуле:
\begin{equation} \label{indirectError}
    \Delta f = \left | \frac{\partial f(x_1, x_2, \ldots, x_n)}{\partial x_1} \right |_{x_2, \ldots, x_n} \Delta x_1 + \ldots + \left | \frac{\partial f(x_1, x_2, \ldots, x_n)}{\partial x_n} \right |_{x_1, \ldots, x_{n-1}} \Delta x_n,
\end{equation}
где $\left | \frac{\partial f(x_1, x_2, \ldots, x_n)}{\partial x_1} \right |_{x_2, \ldots, x_n}$~---~частная производная по $x_1$ при постоянных $x_2, \ldots, x_n$; $\Delta x_1 \ldots \Delta x_n$~---~абсолютные погрешности прямых измерений.

\textbf{Пример.}
Студент хочет измерить мощность силы, которая заставляет кирпич равномерно двигаться вверх по наклонной плоскости. Проведя теоретические расчёты студент получил рабочую формулу $N = F\upsilon\cos{\alpha}$, где $F, \upsilon, \alpha$~---~величины, получаемые в результате прямых измерений. Применив формулу \ref{indirectError} найдём частные производные:
\begin{gather*}
    \left | \frac{\partial N(F, \upsilon, \alpha)}{\partial F} \right |_{\upsilon, \alpha} = \upsilon\cos{\alpha} ; \\
    \left | \frac{\partial N(F, \upsilon, \alpha)}{\partial \upsilon} \right |_{F, \alpha} = F\cos{\alpha} ; \\
    \left | \frac{\partial N(F, \upsilon, \alpha)}{\partial \alpha} \right |_{F, \upsilon} = F\upsilon\sin{\alpha}.
\end{gather*}
И тогда абсолютная погрешность мощности будет определяться по формуле:
\begin{equation*}
    \Delta N = \upsilon\Delta F\cos{\alpha} + F\Delta \upsilon\cos{\alpha} + F\upsilon\Delta\alpha\sin{\alpha},
\end{equation*}
где $\Delta\alpha$ измеряется в радианах ($1^\circ \sim \pi/180$). 

Пусть в результате эксперимента и из паспортных данных были получены следующие значения:

\begin{tabular}{|c|c|c|c|c|c|}
    \hline
    $F$, Н & $\Delta F$, Н & $\upsilon$, см/с & $\Delta \upsilon$, см/с & $\alpha$, рад & $\Delta \alpha$, рад\\ \hline
    $7$ & $0,1$ & $5$ & $0,2$ & $\frac{\pi}{6}$ & $0,002\pi$ \\ \hline
\end{tabular}

И тогда можно посчитать значение абсолютной погрешности косвенных измерений:
\begin{equation}
    \Delta N = 5 \cdot 0,1 \cos{\frac{\pi}{6}} + 7\cdot 0,2\cos{\frac{\pi}{6}} + 7\cdot 5 \cdot 0,002\pi\sin{\frac{\pi}{6}} \approx 1,68\;\text{Вт}.
\end{equation}
\progress{}
\section{}
Используя формулу \ref{indirectError} получить формулу абсолютной погрешности для следующих физических зависимостей:
$H(I,R,\alpha) = \frac{NI}{2R\cdot tg\alpha}$, $Z(R,\omega, C) = \sqrt{R^2+\left (\cfrac{1}{\omega C} \right )^2}$. \\
\section{}
\begin{enumerate}
    \item Создать таблицу. Верхнюю строчку отвести под название выбранной лабораторной работы (Например: <<Определение плотности образца>>, <<Определение  коэффициента вязкости жидкости  методом Стокса>>, <<Определение радиуса кривизны вогнутой поверхности методом катающегося шара>> );
    \item Заполнить первые колонки (не менее трёх колонок) результатами \textbf{прямых} измерений.
    \item Составьте столбец из косвенно измеренных значений $f(x_1, x_2, \ldots, x_n)$;
    \item Найти формулу погрешности косвенных измерений $\Delta f(x_1, x_2, \ldots, x_n)$ используя уравнение \ref{indirectError} и $f(x_1, x_2, \ldots, x_n)$. 
    \item В отдельных колонках по формулам рассчитать абсолютную и относительную погрешность косвенного измерения ($\delta  = \frac{\Delta f(x_1, x_2, \ldots, x_n)}{f(x_1, x_2, \ldots, x_n)}$);
\end{enumerate}

\questions{}
\begin{enumerate}
    \item Что такое аддитивная и мультипликативная погрешность измерения?
    \item Что такое погрешность косвенных измерений? Как находят эту погрешность?
    \item Что такое совместные и совокупные измерения? Приведите примеры.
\end{enumerate}

\textbf{Методические рекомендации к заданию для обучающихся:} 

Результаты выполнения задания оформляются в тетради, или на листочке, а также в файле с расширением <<.xlsx>>
% \textbf{Методические рекомендации к заданию для обучающихся:} 

Выполнение практического задания проводится обучающимся
самостоятельно. Для расчетов используются результаты собственных
исследований полученных в ходе выполнения лабораторных работ по курсу <<механика>> (РЕКОМЕНДУЕТСЯ)
% , при отсутствии необходимого материала данные предоставляются преподавателем по запросу студента
. 
Работа выполняется в программе \href{https://mega.nz/folder/0EUyAKgC#4X5RubnYkoUUJYhwDDpKbg}{Microsoft Office Excel} или \href{https://www.libreoffice.org/download/download-libreoffice/?type=win-x86_64&version=7.4.0&lang=ru-RU}{LibreOffice Calc}. Результаты выполнения задания оформляются в файле с расширением ".xlsx". В названии файла укажите свою фамилию с инициалами, курс и группу.
\end{document}