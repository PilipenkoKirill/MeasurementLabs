%!TEX program = xelatex.exe
\documentclass[14pt,a4paper]{article}
% \usepackage[pdftex,
%     pdfauthor={К.С.~Пилипенко},
%     pdfsubject={The Subject},
%     pdfkeywords={Первое ключевое слово, второе ключевое слово},
%     pdfproducer={LuaLatex with hyperref},
%     pdfcreator={Lualatex},
%     % hidelinks
% ]{hyperref}
%%%%%%%%%%%%%%Пользовательские команды%%%%%%%%%
\usepackage[euler]{textgreek}
\usepackage{calc}
\usepackage{color}
\usepackage{listing}
\usepackage{svg}
\usepackage{hhline}
\usepackage{multirow}
\usepackage{latexsym,amsmath,amssymb,amsbsy,graphicx}
\usepackage{listings}
\usepackage{icomma}
\usepackage[obeyspaces]{url} %Позволяет прописывать путь к файлам
\usepackage[american,siunitx]{circuitikz}
\def\centerarc[#1][#2](#3:#4:#5){\draw[#1]($(#2)+({#5*cos(#3)},{#5*sin(#3)})$)}
\usepackage[version=4]{mhchem} % the canonical chemistry package (example: \ce{^{32}_{15}P})
\usepackage{graphicx}
\usepackage{pgfplots}
\usepackage{wrapfig}
\graphicspath{{images/}}
\DeclareGraphicsExtensions{.pdf,.png,.jpg}
\usepackage{hyperref}
\hypersetup{
    colorlinks=true,
    linkcolor=blue,
    filecolor=magenta,      
    urlcolor=cyan,
    pdfauthor={К.С.~Пилипенко},
    pdfsubject={Основы обработки результатов исследования в физике},
    % pdfkeywords={Первое ключевое слово, второе ключевое слово},
    pdfproducer={LuaLatex with hyperref},
    pdfcreator={Lualatex},
    pdfpagemode=FullScreen,
    }
\urlstyle{same}
%Форматирование разделов
\usepackage{titlesec}
\usepackage{multicol} %Многоколоночный список
%%% Кастомизация разделов
\titleformat
{\section} % command
[block] % shape
{\normalfont\bfseries\itshape} % format
{Задание №\thesection.}{0.5em}{} % label
\newcommand{\progress}{\begin{center}\large\bfseries Ход работы\end{center}}
\newcommand{\questions}{\begin{center}\large\bfseries  Контрольные вопросы \end{center}}
%%%%%%%%%%%%%%%%%%%%%%%%Оформление по ГОСТУ
\usepackage{fontspec}
\setmainfont[Renderer=Basic,Ligatures={TeX}]{Times New Roman}
% \usepackage[english,russian]{babel} %Поддержка русской локализации
\setmonofont{dejavusansmono} % verbatim с кириллицей
\usepackage[14pt]{extsizes} % для того чтобы задать нестандартный 14-ый размер шрифта
\usepackage{indentfirst} %Задаёт отступ самого первого абзаца
\setlength\parindent{1.25cm}
\usepackage[a4paper, left=3cm, top=1.5cm, right=1.5cm, bottom=2cm]{geometry}
\usepackage{setspace}
\sloppy %Выравнивание текст по ширине и решение проблемы переполнением строки
\onehalfspacing %Полуторный интервал
%%%%%%%%%%%%%%%%%%%%%%%%%%%%% Добавление списка литературы
\usepackage{polyglossia}[2014/05/21]            % Поддержка многоязычности (fontspec подгружается автоматически)
\setmainlanguage[babelshorthands=true]{russian}  % Язык по-умолчанию русский с поддержкой приятных команд пакета babel
\setotherlanguage{english}                       % Дополнительный язык = английский (в американской вариации по-умолчанию)
\setmonofont{Courier New}
\newfontfamily\cyrillicfonttt{Courier New}
    \defaultfontfeatures{Ligatures=TeX,Mapping=tex-text}
\setmainfont{Times New Roman}
\newfontfamily\cyrillicfont{Times New Roman}
\setsansfont{Arial}
\newfontfamily\cyrillicfontsf{Arial}
\usepackage[
    bibencoding=utf8,% кодировка bib файла
    sorting=none,% настройка сортировки списка литературы
    style=gost-numeric,% стиль цитирования и библиографии (по ГОСТ)
    language=autobib,% получение языка из babel/polyglossia, default: autobib % если ставить autocite или auto, то цитаты в тексте с указанием страницы, получат указание страницы на языке оригинала
    autolang=other,% многоязычная библиография
    clearlang=true,% внутренний сброс поля language, если он совпадает с языком из babel/polyglossia
    defernumbers=true,% нумерация проставляется после двух компиляций, зато позволяет выцеплять библиографию по ключевым словам и нумеровать не из большего списка
    sortcites=true,% сортировать номера затекстовых ссылок при цитировании (если в квадратных скобках несколько ссылок, то отображаться будут отсортированно, а не абы как)
    doi=false,% Показывать или нет ссылки на DOI
    isbn=false,% Показывать или нет ISBN
]{biblatex}
\DeclareSourcemap{ %модификация bib файла перед тем, как им займётся biblatex 
    \maps{
        \map{% перекидываем значения полей language в поля langid, которыми пользуется biblatex
            \step[fieldsource=language, fieldset=langid, origfieldval, final]
            \step[fieldset=language, null]
        }
    }
}
\emergencystretch=1em
\addbibresource{literature.bib}
%%%%%%%%%%%%%%%%%%%%%%%%
\author{\href{mailto:www-kirill.pilipenko@yandex.ru}{К.С.~Пилипенко} \href{https://github.com/PilipenkoKirill/MeasurementLabs}{\includegraphics[width=.5cm]{images/gitHubLogo.pdf}}} %Через \and можно добавить ещё авторов
\date{\the\year{}}

\title{Лабораторная работа №1 \\ \textit{Характеристика выборки. Дисперсия. Стандартное отклонение. Коэффициент вариации. Ошибка опыта.}}
\begin{document}
\maketitle
\textbf{Среднее арифметическое ($\bar{x}$) выборочной совокупности}:
\begin{equation}
    \bar{x} = \sum\limits^n_{i=1}\frac{x_i}{n},
\end{equation}
где $n$~---~объём выборки (общее количество элементов выборки)

\textbf{Дисперсия случайной величины (D[X])} — мера разброса данной случайной величины, то есть её отклонения от среднего арифметического
\begin{equation}
    D[X] =\cfrac{\sum_{i=1}^n(x_i-\bar{x})^2}{n-1}
\end{equation}

\textbf{Стандартное отклонение ($\sigma$)} :
\begin{equation}
    \sigma = \sqrt{D[X]}  = \sqrt{\cfrac{\sum_{i=1}^n(x_i-\bar{x})^2}{n-1}}
\end{equation}

\textbf{Ошибка средней арифметической (SE)} является мерой отклонения выборочной средней от средней всей (генеральной) совокупности $\mu$.
\begin{equation}
    SE = \sqrt{\frac{D[X]_\text{г.с.}}{n}}
\end{equation}
где $D[X]_\text{г.с.}$~---~величина дисперсии генеральной совокупности

\textbf{Коэффициент вариации V (относительное стандартное отклонение)} — это стандартная мера дисперсии распределения вероятностей или частотного распределения (выражается в процентах)
\begin{equation}
    V = \frac{\sigma}{\bar{x}}\cdot 100\%.
\end{equation}
Градация коэффициента вариации:
\begin{itemize}
    \item < 5\% - очень слабая
    \item 5-10\% -слабая
    \item 10-20\%- средняя
    \item 20-30\% - сильная
    \item > 30\% - очень сильная
\end{itemize}

\textbf{Мода (Mo)} — это наиболее часто встречающееся значение случайной величины.

\textbf{Медиана (Ме)} — число, которое делит ранжированную выборку на две равного объёма, то есть находиться посередине. Если объём выборки – четное число, то медиана равна среднему арифметическому между двумя средними числами выборки.

\progress{}
\begin{enumerate}
    \item Верхнюю строчку отвести под название выбранной лабораторной работы;
    \item Составить таблицу, в первой колонке которой будет записан результат серии наблюдений (от 10 значений), который мог быть получен путём как прямых измерений, так и косвенных;
    \item Устранить из выборки очевидные промахи (при наличии);
    \item Отдельно вывести место под константы и истинное значение (при наличии);
    \item Провести ранжирование данных в первой колонке средствами Excel;
    \item Рассчитать в таблице следующие показатели средствами Excel: \\
    \begin{itemize}
        \item $\bar{x}$~---~среднюю арифметическую,
        \item $\Delta x$~---~абсолютную погрешность измерения (смещение),
        \item $\delta_x$~---~относительную погрешность,
        \item $R$~---~размах,
        \item $D[X]$~---~дисперсию, 
        \item $\sigma$~---~стандартное (среднеквадратическое) отклонение, 
        \item $SE$~---~ошибку средней арифметической (при наличии истинного значения), 
        \item $V$~---~коэффициент вариации, дать оценку вариабельности, 
        % \item sх \%~---~относительную ошибка выборочной средней, 
        \item $Mo$~---~моду,
        \item $Me$~---~медиану.
    \end{itemize}
    \item Адаптировать таблицу под любой, произвольный набор значений
\end{enumerate}
\questions{}
\begin{enumerate}
    \item Как следует записывать результат измерения физической величины?
    \item По какой формуле можно определить среднее значение случайной величины?
    \item Что называют промахом в результатах наблюдения? 
    \item Что такое доверительная вероятность Р? Какое значение доверительной вероятности обычно используют? %0,95 и 0,99 в особых случаях
    \item Почему относительная погрешность является информативней, чем абсолютная?
    % \item Какой смысл дисперсии?
\end{enumerate} 
\textbf{Методические рекомендации к заданию для обучающихся:} 

Выполнение практического задания проводится обучающимся
самостоятельно. Для расчетов используются результаты собственных
исследований полученных в ходе выполнения лабораторных работ по курсу <<механика>> (РЕКОМЕНДУЕТСЯ)
% , при отсутствии необходимого материала данные предоставляются преподавателем по запросу студента
. 
Работа выполняется в программе \href{https://mega.nz/folder/0EUyAKgC#4X5RubnYkoUUJYhwDDpKbg}{Microsoft Office Excel} или \href{https://www.libreoffice.org/download/download-libreoffice/?type=win-x86_64&version=7.4.0&lang=ru-RU}{LibreOffice Calc}. Результаты выполнения задания оформляются в файле с расширением ".xlsx". В названии файла укажите свою фамилию с инициалами, курс и группу.
\end{document}