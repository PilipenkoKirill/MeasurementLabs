%!TEX program = xelatex.exe
\documentclass[14pt,a4paper]{article}
% \usepackage[pdftex,
%     pdfauthor={К.С.~Пилипенко},
%     pdfsubject={The Subject},
%     pdfkeywords={Первое ключевое слово, второе ключевое слово},
%     pdfproducer={LuaLatex with hyperref},
%     pdfcreator={Lualatex},
%     % hidelinks
% ]{hyperref}
%%%%%%%%%%%%%%Пользовательские команды%%%%%%%%%
\usepackage[euler]{textgreek}
\usepackage{calc}
\usepackage{color}
\usepackage{listing}
\usepackage{svg}
\usepackage{hhline}
\usepackage{multirow}
\usepackage{latexsym,amsmath,amssymb,amsbsy,graphicx}
\usepackage{listings}
\usepackage{icomma}
\usepackage[obeyspaces]{url} %Позволяет прописывать путь к файлам
\usepackage[american,siunitx]{circuitikz}
\def\centerarc[#1][#2](#3:#4:#5){\draw[#1]($(#2)+({#5*cos(#3)},{#5*sin(#3)})$)}
\usepackage[version=4]{mhchem} % the canonical chemistry package (example: \ce{^{32}_{15}P})
\usepackage{graphicx}
\usepackage{pgfplots}
\usepackage{wrapfig}
\graphicspath{{images/}}
\DeclareGraphicsExtensions{.pdf,.png,.jpg}
\usepackage{hyperref}
\hypersetup{
    colorlinks=true,
    linkcolor=blue,
    filecolor=magenta,      
    urlcolor=cyan,
    pdfauthor={К.С.~Пилипенко},
    pdfsubject={Основы обработки результатов исследования в физике},
    % pdfkeywords={Первое ключевое слово, второе ключевое слово},
    pdfproducer={LuaLatex with hyperref},
    pdfcreator={Lualatex},
    pdfpagemode=FullScreen,
    }
\urlstyle{same}
%Форматирование разделов
\usepackage{titlesec}
\usepackage{multicol} %Многоколоночный список
%%% Кастомизация разделов
\titleformat
{\section} % command
[block] % shape
{\normalfont\bfseries\itshape} % format
{Задание №\thesection.}{0.5em}{} % label
\newcommand{\progress}{\begin{center}\large\bfseries Ход работы\end{center}}
\newcommand{\questions}{\begin{center}\large\bfseries  Контрольные вопросы \end{center}}
%%%%%%%%%%%%%%%%%%%%%%%%Оформление по ГОСТУ
\usepackage{fontspec}
\setmainfont[Renderer=Basic,Ligatures={TeX}]{Times New Roman}
% \usepackage[english,russian]{babel} %Поддержка русской локализации
\setmonofont{dejavusansmono} % verbatim с кириллицей
\usepackage[14pt]{extsizes} % для того чтобы задать нестандартный 14-ый размер шрифта
\usepackage{indentfirst} %Задаёт отступ самого первого абзаца
\setlength\parindent{1.25cm}
\usepackage[a4paper, left=3cm, top=1.5cm, right=1.5cm, bottom=2cm]{geometry}
\usepackage{setspace}
\sloppy %Выравнивание текст по ширине и решение проблемы переполнением строки
\onehalfspacing %Полуторный интервал
%%%%%%%%%%%%%%%%%%%%%%%%%%%%% Добавление списка литературы
\usepackage{polyglossia}[2014/05/21]            % Поддержка многоязычности (fontspec подгружается автоматически)
\setmainlanguage[babelshorthands=true]{russian}  % Язык по-умолчанию русский с поддержкой приятных команд пакета babel
\setotherlanguage{english}                       % Дополнительный язык = английский (в американской вариации по-умолчанию)
\setmonofont{Courier New}
\newfontfamily\cyrillicfonttt{Courier New}
    \defaultfontfeatures{Ligatures=TeX,Mapping=tex-text}
\setmainfont{Times New Roman}
\newfontfamily\cyrillicfont{Times New Roman}
\setsansfont{Arial}
\newfontfamily\cyrillicfontsf{Arial}
\usepackage[
    bibencoding=utf8,% кодировка bib файла
    sorting=none,% настройка сортировки списка литературы
    style=gost-numeric,% стиль цитирования и библиографии (по ГОСТ)
    language=autobib,% получение языка из babel/polyglossia, default: autobib % если ставить autocite или auto, то цитаты в тексте с указанием страницы, получат указание страницы на языке оригинала
    autolang=other,% многоязычная библиография
    clearlang=true,% внутренний сброс поля language, если он совпадает с языком из babel/polyglossia
    defernumbers=true,% нумерация проставляется после двух компиляций, зато позволяет выцеплять библиографию по ключевым словам и нумеровать не из большего списка
    sortcites=true,% сортировать номера затекстовых ссылок при цитировании (если в квадратных скобках несколько ссылок, то отображаться будут отсортированно, а не абы как)
    doi=false,% Показывать или нет ссылки на DOI
    isbn=false,% Показывать или нет ISBN
]{biblatex}
\DeclareSourcemap{ %модификация bib файла перед тем, как им займётся biblatex 
    \maps{
        \map{% перекидываем значения полей language в поля langid, которыми пользуется biblatex
            \step[fieldsource=language, fieldset=langid, origfieldval, final]
            \step[fieldset=language, null]
        }
    }
}
\emergencystretch=1em
\addbibresource{literature.bib}
%%%%%%%%%%%%%%%%%%%%%%%%
\author{\href{mailto:www-kirill.pilipenko@yandex.ru}{К.С.~Пилипенко} \href{https://github.com/PilipenkoKirill/MeasurementLabs}{\includegraphics[width=.5cm]{images/gitHubLogo.pdf}}} %Через \and можно добавить ещё авторов
\date{\the\year{}}

\title{Лабораторная работа №10 \\ \textit{Статистическая обработка данных в MathCad}}
\begin{document}
\maketitle
Для того чтобы задать то или иное распределение в Mathcad можно воспользоваться функциями из разделов: \texttt{Probability Density} и \texttt{Probability Distribution}. В первом разделе список функций плотности вероятности $\omega(x)$, а во втором --- функции распределения $f(x)$. 

Основные дискретные распределения: 
\begin{enumerate}
	\item Биномиальное (Бернулли): \texttt{dbinom(k,n,p)} и \texttt{pbinom(k,n,p)}, где \texttt{k}~---~количество событий A, \texttt{n}~---~общее количество событий, \texttt{p}~---~вероятность события A;
	\item Геометрическое: \texttt{dgeom($\eta$,p)} и \texttt{pgeom($\eta$,p)}, где $\eta$~---~число испытаний до первого успеха, \texttt{p}~---~вероятность успеха;
	\item Пуассоновское: \texttt{dpois(k,$\bar{n}$)} и \texttt{ppois(k,$\bar{n}$)}.
\end{enumerate}
Основные непрерывные распределения: 
\begin{enumerate}
	\item Равномерное: \texttt{dunif(x,a,b)} и \texttt{punif(x,a,b)}, где \texttt{a,b}~---~границы отрезка [a,b];
	\item Экспоненциальное (показательное): \texttt{dexp(x,$\lambda$)} и \texttt{pexp(x,$\lambda$)};
	\item Нормальное (Гауссовское): \texttt{dnorm(x,$\bar{x}, \sigma$)} и \texttt{pnorm(x,$\bar{x}, \sigma$)};
	\item Распределение $chi^2$: \texttt{dchisq(x,n)} и \texttt{pchisq(x,n)}, где \texttt{n}~---~количество степеней свободы;
	\item Стьюдента:  \texttt{dt(x,n)} и \texttt{pt(x,n)}, где \texttt{n}~---~количество степеней свободы.
	\item F-распределение Фишера: \texttt{dunif(x,n,m)} и \texttt{punif(x,n,m)}, где \texttt{n,m}~---~степени свободы для двух независимых СВ.
\end{enumerate}
\progress{}
\section{Построение распределений дискретных случайных величин}
% Для указанных значений параметров вычислите и постройте графически биномиальное, геометрическое и пуассоновское распределение. Вычислить вероятность попадания СВ в указанный интервал.   
\noindent\begin{enumerate}
	\item Построите биномиальное распределение для серии из 20 независимых испытаний с вероятностью успеха $p = 0,6$. Постройте графики плотности распределения и функции распределения. Найдите значение k, для которой $P(X=k)$ максимальна. Вычислите вероятность попадания случайной величины в интервал (1,5);
	\item Постройте графики плотности распределения и функции распределения для геометрического распределения, если $p=0,4$,  $\eta \in [0,20]$. Найдите наиболее вероятное значение $\eta_\text{наиб.}$. Вычислите вероятность попадания случайной величины в интервал (1,5);
	\item Проделайте все то же самое, что и пункте 1 и 2 для распределения Пуассона с параметрами $\bar{n}=5$, $k \in [0,20]$. 
\end{enumerate}

% \section{Построение распределений непрерывных случайных величин}
% Постройте графики плотности распределения и функции распределения Стьюдента с $n = 2, 5, 10$ степеням свободы и нормального распределения ($\bar{x} = 0, \sigma=1$). 

\section{Построение гистограмм на основе эмпирического распределения}
\noindent \begin{enumerate}
	\item Получить массив значений используя функцию \texttt{rnorm(k, \textmu, \textsigma)};
	\item Найти максимальное, минимальное значения и размах выборки пользуясь функциями \texttt{max} и \texttt{min};
	\item %TODO:
\end{enumerate}

\questions{}
Приведите примеры распределений
\begin{enumerate}
	\item Биномиального;
	\item Геометрического;
	\item Равномерного;
	\item Экспоненциального.
\end{enumerate}

\nocite{Plis2003, Krestlev2010,MathCADshortcuts,Ochkov2016}
\printbibliography[title={Рекомендуемая литература}]
\end{document}