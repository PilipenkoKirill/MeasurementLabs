%!TEX program = xelatex.exe
\documentclass[14pt,a4paper]{article}
% \usepackage[pdftex,
%     pdfauthor={К.С.~Пилипенко},
%     pdfsubject={The Subject},
%     pdfkeywords={Первое ключевое слово, второе ключевое слово},
%     pdfproducer={LuaLatex with hyperref},
%     pdfcreator={Lualatex},
%     % hidelinks
% ]{hyperref}
%%%%%%%%%%%%%%Пользовательские команды%%%%%%%%%
\usepackage[euler]{textgreek}
\usepackage{calc}
\usepackage{color}
\usepackage{listing}
\usepackage{svg}
\usepackage{hhline}
\usepackage{multirow}
\usepackage{latexsym,amsmath,amssymb,amsbsy,graphicx}
\usepackage{listings}
\usepackage{icomma}
\usepackage[obeyspaces]{url} %Позволяет прописывать путь к файлам
\usepackage[american,siunitx]{circuitikz}
\def\centerarc[#1][#2](#3:#4:#5){\draw[#1]($(#2)+({#5*cos(#3)},{#5*sin(#3)})$)}
\usepackage[version=4]{mhchem} % the canonical chemistry package (example: \ce{^{32}_{15}P})
\usepackage{graphicx}
\usepackage{pgfplots}
\usepackage{wrapfig}
\graphicspath{{images/}}
\DeclareGraphicsExtensions{.pdf,.png,.jpg}
\usepackage{hyperref}
\hypersetup{
    colorlinks=true,
    linkcolor=blue,
    filecolor=magenta,      
    urlcolor=cyan,
    pdfauthor={К.С.~Пилипенко},
    pdfsubject={Основы обработки результатов исследования в физике},
    % pdfkeywords={Первое ключевое слово, второе ключевое слово},
    pdfproducer={LuaLatex with hyperref},
    pdfcreator={Lualatex},
    pdfpagemode=FullScreen,
    }
\urlstyle{same}
%Форматирование разделов
\usepackage{titlesec}
\usepackage{multicol} %Многоколоночный список
%%% Кастомизация разделов
\titleformat
{\section} % command
[block] % shape
{\normalfont\bfseries\itshape} % format
{Задание №\thesection.}{0.5em}{} % label
\newcommand{\progress}{\begin{center}\large\bfseries Ход работы\end{center}}
\newcommand{\questions}{\begin{center}\large\bfseries  Контрольные вопросы \end{center}}
%%%%%%%%%%%%%%%%%%%%%%%%Оформление по ГОСТУ
\usepackage{fontspec}
\setmainfont[Renderer=Basic,Ligatures={TeX}]{Times New Roman}
% \usepackage[english,russian]{babel} %Поддержка русской локализации
\setmonofont{dejavusansmono} % verbatim с кириллицей
\usepackage[14pt]{extsizes} % для того чтобы задать нестандартный 14-ый размер шрифта
\usepackage{indentfirst} %Задаёт отступ самого первого абзаца
\setlength\parindent{1.25cm}
\usepackage[a4paper, left=3cm, top=1.5cm, right=1.5cm, bottom=2cm]{geometry}
\usepackage{setspace}
\sloppy %Выравнивание текст по ширине и решение проблемы переполнением строки
\onehalfspacing %Полуторный интервал
%%%%%%%%%%%%%%%%%%%%%%%%%%%%% Добавление списка литературы
\usepackage{polyglossia}[2014/05/21]            % Поддержка многоязычности (fontspec подгружается автоматически)
\setmainlanguage[babelshorthands=true]{russian}  % Язык по-умолчанию русский с поддержкой приятных команд пакета babel
\setotherlanguage{english}                       % Дополнительный язык = английский (в американской вариации по-умолчанию)
\setmonofont{Courier New}
\newfontfamily\cyrillicfonttt{Courier New}
    \defaultfontfeatures{Ligatures=TeX,Mapping=tex-text}
\setmainfont{Times New Roman}
\newfontfamily\cyrillicfont{Times New Roman}
\setsansfont{Arial}
\newfontfamily\cyrillicfontsf{Arial}
\usepackage[
    bibencoding=utf8,% кодировка bib файла
    sorting=none,% настройка сортировки списка литературы
    style=gost-numeric,% стиль цитирования и библиографии (по ГОСТ)
    language=autobib,% получение языка из babel/polyglossia, default: autobib % если ставить autocite или auto, то цитаты в тексте с указанием страницы, получат указание страницы на языке оригинала
    autolang=other,% многоязычная библиография
    clearlang=true,% внутренний сброс поля language, если он совпадает с языком из babel/polyglossia
    defernumbers=true,% нумерация проставляется после двух компиляций, зато позволяет выцеплять библиографию по ключевым словам и нумеровать не из большего списка
    sortcites=true,% сортировать номера затекстовых ссылок при цитировании (если в квадратных скобках несколько ссылок, то отображаться будут отсортированно, а не абы как)
    doi=false,% Показывать или нет ссылки на DOI
    isbn=false,% Показывать или нет ISBN
]{biblatex}
\DeclareSourcemap{ %модификация bib файла перед тем, как им займётся biblatex 
    \maps{
        \map{% перекидываем значения полей language в поля langid, которыми пользуется biblatex
            \step[fieldsource=language, fieldset=langid, origfieldval, final]
            \step[fieldset=language, null]
        }
    }
}
\emergencystretch=1em
\addbibresource{literature.bib}
%%%%%%%%%%%%%%%%%%%%%%%%
\author{\href{mailto:www-kirill.pilipenko@yandex.ru}{К.С.~Пилипенко} \href{https://github.com/PilipenkoKirill/MeasurementLabs}{\includegraphics[width=.5cm]{images/gitHubLogo.pdf}}} %Через \and можно добавить ещё авторов
\date{\the\year{}}

\title{Лабораторная работа №11 \\ \textit{Основы регрессионного анализа. Коэффициент корреляции и ковариации}}
\begin{document}
\maketitle

\textbf{Оборудование и принадлежности:} компьютер с установленным программным обеспечением для статистического анализа данных (Excel или MathCAD).

\emph{Регрессионный анализ} — набор статистических методов исследования влияния одной или нескольких независимых переменных $X_{1}, X_{2}, ... , X_{p}$ на зависимую переменную $Y$. \underline{Независимые переменные} иначе называют \emph{регрессорами} или предикторами, а \underline{зависимые переменные} — \emph{критериальными} или регрессантами. Наиболее распространённый вид регрессионного анализа — линейная регрессия, когда находят линейную функцию, которая, согласно определённым математическим критериям, наиболее соответствует данным. Например, в методе наименьших квадратов вычисляется прямая, сумма квадратов между которой и данными минимальна. 

\progress{}

    Ознакомьтесь с понятиями корреляции и ковариации. Определите их значения для двух случайных величин X и Y, используя следующие формулы:

    Ковариация: cov(X,Y) = E[(X - E[X])(Y - E[Y])]
    Коэффициент корреляции: r(X,Y) = cov(X,Y) / (SD[X] * SD[Y])

    Загрузите набор данных для анализа. Выберите две количественные переменные из набора данных, которые, по вашему мнению, могут быть связаны друг с другом (например, возраст и доход).

    Рассчитайте ковариацию и коэффициент корреляции для этих двух переменных, используя программное обеспечение для статистического анализа данных. Объясните, что означают полученные значения.

    Визуализируйте данные с помощью диаграммы рассеяния. Определите, есть ли между переменными линейная зависимость.

    Измените данные, чтобы увидеть, как это повлияет на значения ковариации и коэффициента корреляции. Попробуйте изменить степень зависимости между переменными, добавив или удалив выбросы, изменяя диапазоны значений или меняя распределение.

    Интерпретируйте полученные результаты. Обсудите, как значение коэффициента корреляции и ковариации связано с характером зависимости между переменными и как эти показатели могут использоваться для прогнозирования значений одной переменной на основе другой.

    Оформите отчет по работе, включающий описание процесса анализа данных, полученные результаты и их интерпретацию.

% \nocite{Plis2003, Krestlev2010,MathCADshortcuts,Ochkov2016}
% \printbibliography[title={Рекомендуемая литература}]
\begin{center}
	\textbf{\Large Приложение}
\end{center}
\lstset{ %
	language=[Visual]Basic,
	keywordstyle=\color{brown},          % Стиль ключевых слов
	breaklines=true,
	numbers=left, 
	backgroundcolor=\color{lightgray!5!white},
	tabsize=4,
	label=SqDistGen, 
	caption={Код генератора выборок для регрессионного анализа}
	} 
	\renewcommand{\lstlistingname}{Листинг}
\begin{lstlisting}
	Dim i As Long
	Dim a As Double
	Dim b As Double
	Dim c As Double
	Dim Error As Double
	Dim random As Double

	a = 2 * (2 * Rnd - 1)
	b = 3 * (2 * Rnd - 1)
	c = 2 * (2 * Rnd - 1)
	Error = 0.2
	i = 51
	Range("A1").Select
	For i = 2 To i
		ActiveCell.Value = i
		ActiveCell.Offset(0, 1).Select
		random = a * i * i + b * i + c
		ActiveCell.Value = random * (1 - Error + 2 * Error * Rnd)
		ActiveCell.Offset(1, -1).Select
	Next i
	\end{lstlisting}
\end{document}