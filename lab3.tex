%!TEX program = xelatex.exe
\documentclass[14pt,a4paper]{article}
% \usepackage[pdftex,
%     pdfauthor={К.С.~Пилипенко},
%     pdfsubject={The Subject},
%     pdfkeywords={Первое ключевое слово, второе ключевое слово},
%     pdfproducer={LuaLatex with hyperref},
%     pdfcreator={Lualatex},
%     % hidelinks
% ]{hyperref}
%%%%%%%%%%%%%%Пользовательские команды%%%%%%%%%
\usepackage[euler]{textgreek}
\usepackage{calc}
\usepackage{color}
\usepackage{listing}
\usepackage{svg}
\usepackage{hhline}
\usepackage{multirow}
\usepackage{latexsym,amsmath,amssymb,amsbsy,graphicx}
\usepackage{listings}
\usepackage{icomma}
\usepackage[obeyspaces]{url} %Позволяет прописывать путь к файлам
\usepackage[american,siunitx]{circuitikz}
\def\centerarc[#1][#2](#3:#4:#5){\draw[#1]($(#2)+({#5*cos(#3)},{#5*sin(#3)})$)}
\usepackage[version=4]{mhchem} % the canonical chemistry package (example: \ce{^{32}_{15}P})
\usepackage{graphicx}
\usepackage{pgfplots}
\usepackage{wrapfig}
\graphicspath{{images/}}
\DeclareGraphicsExtensions{.pdf,.png,.jpg}
\usepackage{hyperref}
\hypersetup{
    colorlinks=true,
    linkcolor=blue,
    filecolor=magenta,      
    urlcolor=cyan,
    pdfauthor={К.С.~Пилипенко},
    pdfsubject={Основы обработки результатов исследования в физике},
    % pdfkeywords={Первое ключевое слово, второе ключевое слово},
    pdfproducer={LuaLatex with hyperref},
    pdfcreator={Lualatex},
    pdfpagemode=FullScreen,
    }
\urlstyle{same}
%Форматирование разделов
\usepackage{titlesec}
\usepackage{multicol} %Многоколоночный список
%%% Кастомизация разделов
\titleformat
{\section} % command
[block] % shape
{\normalfont\bfseries\itshape} % format
{Задание №\thesection.}{0.5em}{} % label
\newcommand{\progress}{\begin{center}\large\bfseries Ход работы\end{center}}
\newcommand{\questions}{\begin{center}\large\bfseries  Контрольные вопросы \end{center}}
%%%%%%%%%%%%%%%%%%%%%%%%Оформление по ГОСТУ
\usepackage{fontspec}
\setmainfont[Renderer=Basic,Ligatures={TeX}]{Times New Roman}
% \usepackage[english,russian]{babel} %Поддержка русской локализации
\setmonofont{dejavusansmono} % verbatim с кириллицей
\usepackage[14pt]{extsizes} % для того чтобы задать нестандартный 14-ый размер шрифта
\usepackage{indentfirst} %Задаёт отступ самого первого абзаца
\setlength\parindent{1.25cm}
\usepackage[a4paper, left=3cm, top=1.5cm, right=1.5cm, bottom=2cm]{geometry}
\usepackage{setspace}
\sloppy %Выравнивание текст по ширине и решение проблемы переполнением строки
\onehalfspacing %Полуторный интервал
%%%%%%%%%%%%%%%%%%%%%%%%%%%%% Добавление списка литературы
\usepackage{polyglossia}[2014/05/21]            % Поддержка многоязычности (fontspec подгружается автоматически)
\setmainlanguage[babelshorthands=true]{russian}  % Язык по-умолчанию русский с поддержкой приятных команд пакета babel
\setotherlanguage{english}                       % Дополнительный язык = английский (в американской вариации по-умолчанию)
\setmonofont{Courier New}
\newfontfamily\cyrillicfonttt{Courier New}
    \defaultfontfeatures{Ligatures=TeX,Mapping=tex-text}
\setmainfont{Times New Roman}
\newfontfamily\cyrillicfont{Times New Roman}
\setsansfont{Arial}
\newfontfamily\cyrillicfontsf{Arial}
\usepackage[
    bibencoding=utf8,% кодировка bib файла
    sorting=none,% настройка сортировки списка литературы
    style=gost-numeric,% стиль цитирования и библиографии (по ГОСТ)
    language=autobib,% получение языка из babel/polyglossia, default: autobib % если ставить autocite или auto, то цитаты в тексте с указанием страницы, получат указание страницы на языке оригинала
    autolang=other,% многоязычная библиография
    clearlang=true,% внутренний сброс поля language, если он совпадает с языком из babel/polyglossia
    defernumbers=true,% нумерация проставляется после двух компиляций, зато позволяет выцеплять библиографию по ключевым словам и нумеровать не из большего списка
    sortcites=true,% сортировать номера затекстовых ссылок при цитировании (если в квадратных скобках несколько ссылок, то отображаться будут отсортированно, а не абы как)
    doi=false,% Показывать или нет ссылки на DOI
    isbn=false,% Показывать или нет ISBN
]{biblatex}
\DeclareSourcemap{ %модификация bib файла перед тем, как им займётся biblatex 
    \maps{
        \map{% перекидываем значения полей language в поля langid, которыми пользуется biblatex
            \step[fieldsource=language, fieldset=langid, origfieldval, final]
            \step[fieldset=language, null]
        }
    }
}
\emergencystretch=1em
\addbibresource{literature.bib}
%%%%%%%%%%%%%%%%%%%%%%%%
\author{\href{mailto:www-kirill.pilipenko@yandex.ru}{К.С.~Пилипенко} \href{https://github.com/PilipenkoKirill/MeasurementLabs}{\includegraphics[width=.5cm]{images/gitHubLogo.pdf}}} %Через \and можно добавить ещё авторов
\date{\the\year{}}

\title{Лабораторная работа №3 \\ \textit{Массивы, генерация массивов, матрицы}}
\begin{document}
\maketitle


\textit{
\begin{center}
    Ход работы:
\end{center}
\section{}
\begin{enumerate}
    \item Сгенерировать массив из 5 целых случайных чисел от 1 до 30, расположить его в ряд таблицы Excel;
    \item Сгенерировать ещё 4 ряда таких же массивов до квадратной матрицы;
    \item Найти определитель этой матрицы;
    \item Произвести транспонировку этой матрицы (1. выделить диапазон 5 х 5, который не пересекается с исходным диапазоном; 2. В строке формул ввести формулу \texttt{ТРАНСП(A1:E5)} и нажать комбинацию клавиш \texttt{CTRL+SHIFT+ENTER}. Также можно выделить матрицу, скопировать её, в параметрах вставки выбрать <<транспонировать>>, но в будущем при генерации новых элементов исходной матрицы они не будут соответствовать транспонированной);
    \item Найти обратную матрицу c помощью функции \texttt{МОБР(диапазон исх. матрицы)} по такому же принципу, как и в пункте выше;
    \item Проверить результат перемножив исходную и обратную матрицы и отобразив отдельно результат.
\end{enumerate}
\section{}
Докажите, что матрица $P$ идемпотентна, то есть для неё выполняется условие $P^2 = P$. 
% Вычислите ее
Покажите, что матрица $I = 2P - E$ инволютивна ($I^2 = E$), здесь $E$~---~единичная матрицы.
\begin{equation}
    P = \begin{pmatrix}
        -26 & -18 & -27 \\
        21 & 15 & 21 \\
        12 & 8 & 13 \\
    \end{pmatrix}.
\end{equation}
}

\questions{}
\begin{enumerate}
    \item Какая функция (набор функций) используется может быть использована для получения случайного числа?
    \item В каком случае матрица называется ортогональной?
    \item По какому алгоритму находят обратную матрицу $A^{-1}$?
\end{enumerate}
% \textbf{Методические рекомендации к заданию для обучающихся:} 

Выполнение практического задания проводится обучающимся
самостоятельно. Для расчетов используются результаты собственных
исследований полученных в ходе выполнения лабораторных работ по курсу <<механика>> (РЕКОМЕНДУЕТСЯ)
% , при отсутствии необходимого материала данные предоставляются преподавателем по запросу студента
. 
Работа выполняется в программе \href{https://mega.nz/folder/0EUyAKgC#4X5RubnYkoUUJYhwDDpKbg}{Microsoft Office Excel} или \href{https://www.libreoffice.org/download/download-libreoffice/?type=win-x86_64&version=7.4.0&lang=ru-RU}{LibreOffice Calc}. Результаты выполнения задания оформляются в файле с расширением ".xlsx". В названии файла укажите свою фамилию с инициалами, курс и группу.
\end{document}